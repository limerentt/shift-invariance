\chapter{Обсуждение результатов} \label{discussion}

В данной главе приводится обобщенный анализ результатов исследования пространственной инвариантности в CNN, обсуждаются ограничения проведенных экспериментов, формулируются практические рекомендации по применению методов анти-алиасинга и намечаются перспективные направления дальнейших исследований.

\section{Синтез экспериментальных результатов}
\label{discussion:synthesis}

Проведенное исследование позволило комплексно оценить феномен пространственной инвариантности в современных архитектурах CNN как для задач классификации, так и для задач детекции объектов. Обобщая результаты всех экспериментов, можно выделить следующие ключевые наблюдения и закономерности:

\subsection{Количественная оценка проблемы инвариантности}
\label{discussion:synthesis:quantification}

Экспериментально подтверждено, что современные CNN демонстрируют значительную чувствительность к субпиксельным сдвигам входных изображений. Для базовых архитектур были получены следующие количественные оценки:

\begin{itemize}
    \item В задачах классификации минимальное косинусное сходство между векторами признаков оригинального и сдвинутого изображения может опускаться до 0.81 для VGG16 и 0.88 для ResNet50.
    
    \item Дрейф уверенности предсказаний при субпиксельных сдвигах достигает 18.8\% для VGG16 и 14.6\% для ResNet50, что может приводить к "переключениям" между классами.
    
    \item В задачах детекции медиана IoU между предсказанными ограничивающими рамками и истинными рамками составляет около 0.68 для YOLOv5, с дрейфом центра рамки до 33.9 пикселей.
\end{itemize}

Эти показатели подтверждают, что проблема недостаточной инвариантности к сдвигам имеет значительный практический эффект, особенно в задачах, требующих высокой точности локализации объектов.

\subsection{Эффективность методов анти-алиасинга}
\label{discussion:synthesis:effectiveness}

Оба исследованных метода анти-алиасинга (BlurPool и TIPS) продемонстрировали высокую эффективность в повышении инвариантности CNN к сдвигам:

\begin{itemize}
    \item Метод BlurPool повышает минимальное косинусное сходство до 0.93 для VGG16 и 0.94 для ResNet50, снижая дрейф уверенности до 7.4\% и 5.2\% соответственно.
    
    \item Метод TIPS обеспечивает наилучшие результаты, с минимальным косинусным сходством около 0.96 для VGG16 и 0.97 для ResNet50, и дрейфом уверенности менее 3\%.
    
    \item В задачах детекции методы анти-алиасинга особенно эффективны: BlurPool повышает медиану IoU до 0.88 и снижает дрейф центра до 8.8 пикселей, а TIPS — до 0.99 и 0.02 пикселя соответственно.
\end{itemize}

Рисунок \ref{fig:model_performance_radar} наглядно демонстрирует преимущества методов анти-алиасинга по всем ключевым метрикам стабильности.

\subsection{Факторы, влияющие на инвариантность}
\label{discussion:synthesis:factors}

Аблационные эксперименты выявили несколько ключевых факторов, влияющих на инвариантность CNN к сдвигам:

\begin{itemize}
    \item \textbf{Размер рецептивного поля}: Более широкое рецептивное поле частично компенсирует чувствительность к сдвигам, но не устраняет основную проблему алиасинга. Даже модели с большим рецептивным полем (>150 пикселей) без анти-алиасинга демонстрируют заметную чувствительность.
    
    \item \textbf{Тип архитектуры}: Архитектуры с остаточными соединениями (ResNet) изначально демонстрируют лучшую инвариантность, чем классические сети без остаточных связей (VGG). Это может быть связано с тем, что остаточные соединения позволяют информации "обходить" операции даунсэмплинга.
    
    \item \textbf{Характеристики фильтра}: Для BlurPool оптимальный баланс между эффективностью и вычислительной сложностью достигается при использовании биномиального фильтра 3-го порядка, хотя фильтры с более плавной характеристикой (биномиальный 4-го порядка, гауссовский) могут давать немного лучшие результаты ценой большей вычислительной сложности.
\end{itemize}

Эти наблюдения согласуются с теоретическими аспектами, рассмотренными в главе \ref{theory}, и подтверждают, что основная причина недостаточной инвариантности связана с алиасингом при операциях даунсэмплинга.

\subsection{Компромисс между производительностью и инвариантностью}
\label{discussion:synthesis:tradeoff}

Профилирование производительности выявило ожидаемый компромисс между инвариантностью и вычислительной эффективностью:

\begin{itemize}
    \item Метод BlurPool снижает FPS на 6.6-9.4\% на ПК и 8.7-12.1\% на встраиваемых системах типа Jetson Xavier NX, обеспечивая значительное улучшение инвариантности.
    
    \item Метод TIPS требует больше вычислительных ресурсов, снижая FPS на 14.8-19.0\% на ПК и 19.7-25.7\% на встраиваемых системах, но обеспечивает почти идеальную инвариантность.
    
    \item Абсолютное увеличение задержки обработки составляет около 0.7 мс для BlurPool и 1.7 мс для TIPS на ПК, что во многих практических сценариях является приемлемым.
\end{itemize}

Данный компромисс можно оптимизировать в зависимости от конкретного применения, как обсуждается в разделе \ref{discussion:recommendations}.

\section{Ограничения исследования}
\label{discussion:limitations}

Хотя проведенное исследование предоставляет убедительные доказательства эффективности методов анти-алиасинга, важно отметить следующие ограничения:

\subsection{Ограничения тестовых данных}
\label{discussion:limitations:data}

\begin{itemize}
    \item Эксперименты проводились на относительно простых данных с контролируемыми сдвигами, тогда как в реальных сценариях объекты могут двигаться сложными траекториями, менять масштаб, форму и освещение.
    
    \item Использовалось ограниченное число фоновых изображений и объектов, что может не полностью отражать разнообразие реальных сцен.
    
    \item Сдвиги были ограничены горизонтальным направлением, хотя в реальных сценариях объекты могут двигаться в произвольных направлениях.
\end{itemize}

\subsection{Ограничения исследованных моделей}
\label{discussion:limitations:models}

\begin{itemize}
    \item Исследовались только определенные архитектуры (VGG16, ResNet50, YOLOv5), результаты могут отличаться для других современных архитектур, особенно для трансформеров и гибридных CNN-Transformer моделей.
    
    \item Модификации с анти-алиасингом не проходили дообучение на задачах, что могло ограничить их потенциал, особенно для метода TIPS, который может выиграть от совместной оптимизации с остальной частью сети.
    
    \item Не исследовалось влияние квантизации и других техник оптимизации моделей, которые могут взаимодействовать с методами анти-алиасинга.
\end{itemize}

\subsection{Методологические ограничения}
\label{discussion:limitations:methodology}

\begin{itemize}
    \item Исследование фокусировалось на инвариантности к сдвигам, не затрагивая подробно другие виды инвариантности (к масштабу, повороту, деформациям).
    
    \item Некоторые метрики (например, косинусное сходство векторов признаков) могут не полностью отражать семантическую близость представлений с точки зрения конечной задачи.
    
    \item Не проводился детальный анализ влияния методов анти-алиасинга на устойчивость к различным типам шума и атак, что может быть важно для практических приложений.
\end{itemize}

Эти ограничения создают возможности для будущих исследований, как обсуждается в разделе \ref{discussion:future}.

\section{Практические рекомендации}
\label{discussion:recommendations}

На основе проведенного исследования можно сформулировать следующие практические рекомендации по применению методов анти-алиасинга в различных сценариях:

\subsection{Рекомендации по выбору метода анти-алиасинга}
\label{discussion:recommendations:method}

\begin{itemize}
    \item \textbf{Для задач классификации без жестких ограничений на производительность}: Рекомендуется применять метод TIPS, обеспечивающий наилучшую инвариантность с приемлемыми вычислительными затратами. Особенно это актуально для приложений, где критично отсутствие "переключений" между классами при малых изменениях входа (медицинская диагностика, системы контроля качества).
    
    \item \textbf{Для задач детекции объектов в видеопотоке}: Применение хотя бы простейших методов анти-алиасинга (BlurPool) критически важно для уменьшения "дрожания" ограничивающих рамок при движении объектов, что особенно важно для трекинга и измерения траекторий.
    
    \item \textbf{Для систем с ограниченными вычислительными ресурсами}: Метод BlurPool представляет собой хороший компромисс между инвариантностью и эффективностью. Для ещё большей оптимизации можно применять его только к наиболее критичным слоям даунсэмплинга (обычно первым 2-3 слоям в сети).
    
    \item \textbf{Для высокоточных систем компьютерного зрения}: В приложениях, требующих максимальной стабильности предсказаний (робототехника, автономные системы, прецизионное управление), рекомендуется использовать метод TIPS, который обеспечивает практически идеальную инвариантность к сдвигам.
\end{itemize}

\subsection{Рекомендации по интеграции в существующие системы}
\label{discussion:recommendations:integration}

\begin{itemize}
    \item \textbf{Локальная модификация критичных слоев}: Вместо полной замены модели можно модифицировать только критичные слои даунсэмплинга, что минимизирует влияние на общую вычислительную сложность и архитектуру системы.
    
    \item \textbf{Постобработка предсказаний}: Для существующих систем, где модификация архитектуры невозможна, можно применять методы постобработки предсказаний (например, усреднение предсказаний для слегка сдвинутых копий входа или временная фильтрация предсказаний для видеопоследовательностей).
    
    \item \textbf{Обучение с аугментациями}: Хотя методы анти-алиасинга решают проблему архитектурно, дополнительное обучение с субпиксельными сдвигами в качестве аугментаций может еще больше повысить устойчивость моделей.
    
    \item \textbf{Комбинирование с другими методами повышения устойчивости}: Методы анти-алиасинга хорошо сочетаются с другими подходами к повышению устойчивости моделей, такими как регуляризация, дистилляция знаний и ансамблирование.
\end{itemize}

\subsection{Отраслевые рекомендации}
\label{discussion:recommendations:industry}

\begin{itemize}
    \item \textbf{Автономные транспортные средства}: Применение методов анти-алиасинга особенно важно для систем компьютерного зрения в автономных транспортных средствах, где нестабильность детекции может приводить к критическим ошибкам при принятии решений.
    
    \item \textbf{Медицинская визуализация}: В системах анализа медицинских изображений инвариантность к малым сдвигам критически важна, поскольку положение органов и патологий может немного варьироваться между изображениями.
    
    \item \textbf{Промышленные системы контроля качества}: Для систем визуального контроля качества на производственных линиях стабильность предсказаний может значительно снизить количество ложных срабатываний.
    
    \item \textbf{Робототехника}: В системах компьютерного зрения для роботов, особенно манипуляторов, стабильность детекции объектов напрямую влияет на точность выполнения операций.
\end{itemize}

\section{Будущие направления исследования}
\label{discussion:future}

Проведенное исследование открывает несколько перспективных направлений для дальнейших работ:

\subsection{Теоретические направления}
\label{discussion:future:theoretical}

\begin{itemize}
    \item \textbf{Разработка новых методов анти-алиасинга}: Хотя TIPS демонстрирует отличные результаты, возможна разработка ещё более эффективных методов, особенно с точки зрения вычислительной сложности.
    
    \item \textbf{Формальный анализ инвариантности более сложных архитектур}: Расширение формального анализа инвариантности на современные архитектуры, включая трансформеры, графовые нейронные сети и другие не-CNN архитектуры.
    
    \item \textbf{Объединение различных типов инвариантности}: Исследование взаимодействия между различными типами инвариантности (к сдвигу, масштабу, повороту) и разработка архитектур, совмещающих все эти свойства.
    
    \item \textbf{Анализ с точки зрения информационной теории}: Более глубокое исследование феномена инвариантности с позиций информационной теории, включая анализ потери информации при операциях даунсэмплинга и возможности её минимизации.
\end{itemize}

\subsection{Прикладные направления}
\label{discussion:future:applied}

\begin{itemize}
    \item \textbf{Применение к другим типам данных}: Исследование эффективности методов анти-алиасинга для других типов данных, таких как трехмерные изображения, видео, медицинские изображения различных модальностей.
    
    \item \textbf{Интеграция с нейросетевыми архитектурами нового поколения}: Адаптация методов анти-алиасинга для современных гибридных архитектур, таких как ConvNext, Vision Transformer, Swin Transformer.
    
    \item \textbf{Оптимизация вычислительной эффективности}: Разработка более эффективных реализаций методов анти-алиасинга, включая применение квантизации, прунинга и других техник оптимизации моделей.
    
    \item \textbf{Исследование в контексте специализированных аппаратных ускорителей}: Оптимизация методов анти-алиасинга для специализированных нейросетевых ускорителей, таких как Google TPU, нейроморфные чипы и другие.
\end{itemize}

\subsection{Междисциплинарные направления}
\label{discussion:future:interdisciplinary}

\begin{itemize}
    \item \textbf{Инспирированные биологией подходы}: Исследование того, как биологические системы зрения решают проблему инвариантности, и применение этих принципов к искусственным нейронным сетям.
    
    \item \textbf{Связь с робастностью к атакам}: Исследование взаимосвязи между инвариантностью к сдвигам и устойчивостью к различным типам атак, таким как состязательные атаки (adversarial attacks).
    
    \item \textbf{Обобщаемость и трансферное обучение}: Изучение влияния улучшенной инвариантности на способность моделей к обобщению и трансферному обучению в условиях ограниченных данных.
    
    \item \textbf{Интерпретируемость и объяснимость}: Исследование того, как повышенная инвариантность влияет на интерпретируемость и объяснимость моделей глубокого обучения.
\end{itemize}

Эти направления исследований могут привести к значительному прогрессу в понимании и улучшении пространственной инвариантности нейронных сетей, что критически важно для их надежного применения в разнообразных практических задачах. 