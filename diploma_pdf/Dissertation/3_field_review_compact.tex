\chapter{Обзор литературы} \label{review}

\section{Инвариантность к сдвигу в CNN-классификаторах}
\label{review:invariance}

Сверточные нейронные сети (CNN) в теории должны обладать определенной степенью инвариантности к позиционным сдвигам входных данных. Это свойство изначально заложено в их архитектуру через механизм разделения весов и локальные рецептивные поля \cite{LeCun1998}. Однако, как показывают многочисленные исследования последних лет, современные CNN демонстрируют ограниченную инвариантность к сдвигам.

\subsection{Теоретические основы и эмпирические исследования}
\label{review:invariance:theory}

Проблема инвариантности в CNN может быть систематически исследована через призму классической теории обработки сигналов. Azulay and Weiss \cite{Azulay2018} показали, что отсутствие антиалиасинговых фильтров перед операциями субдискретизации приводит к высокочастотному шуму в представлениях признаков, что делает модель чувствительной к малым сдвигам входных данных.

Zhang \cite{Zhang2019} идентифицировал операции даунсэмплинга (в частности, max-pooling и свертку с шагом больше 1) как основной источник нарушения инвариантности к сдвигам. В этой работе было показано, что субпиксельные сдвиги входных изображений могут приводить к значительным изменениям в активациях нейронов и, как следствие, к нестабильности выходных предсказаний модели.

Исследования показывают, что более глубокие сети, такие как ResNet \cite{He2016}, как правило, более инвариантны к сдвигам по сравнению с менее глубокими архитектурами, такими как AlexNet или VGG \cite{Simonyan2014}. Также наблюдается влияние типа пулинга на инвариантность: average-pooling обеспечивает лучшую инвариантность к сдвигам по сравнению с max-pooling.

\section{Методы анти-алиасинга в нейронных сетях}
\label{review:antialias}

\subsection{Методы улучшения инвариантности: BlurPool и адаптивные подходы}
\label{review:antialias:blurpool_and_adaptive}

После идентификации алиасинга как основной причины нарушения инвариантности к сдвигам в CNN, исследователи предложили ряд методов для решения этой проблемы.

\subsection{Низкочастотная фильтрация и BlurPool}
\label{review:antialias:blurpool}

Наиболее прямолинейным подходом к борьбе с алиасингом является применение низкочастотной фильтрации перед операциями субдискретизации. Этот подход был впервые систематически применен к CNN в работе Zhang \cite{Zhang2019}, где был предложен метод BlurPool (Blur-then-downsampling).

В BlurPool операции max-pooling и свертки с шагом больше 1 модифицируются таким образом, что перед непосредственной субдискретизацией применяется размытие с использованием фиксированного низкочастотного фильтра. Авторы исследовали различные типы фильтров, включая простое усреднение (box filter), треугольный фильтр (binomial filter) и фильтр Гаусса, и показали, что даже простейшие из них значительно улучшают инвариантность сети к сдвигам.

Математически модификация стандартных операций может быть представлена следующим образом:

Для MaxPool с ядром $k$ и шагом $s$:
\begin{equation}
\text{MaxPool}_{k,s} = \text{Subsample}_{s} \circ \text{Max}_{k,1} \longrightarrow \text{Subsample}_{s} \circ \text{Blur}_{g} \circ \text{Max}_{k,1}
\end{equation}

Для свертки с шагом $s$:
\begin{equation}
\text{Conv}_{k,s} \longrightarrow \text{BlurPool}_{m,s} \circ \text{ReLU} \circ \text{Conv}_{k,1}
\end{equation}

где $g$ — низкочастотный фильтр размером $m \times m$, а символ $\circ$ обозначает композицию операций.

Важнейшие свойства низкочастотного фильтра:
\begin{itemize}
    \item Фильтр должен иметь нормированные веса: $\sum g_{ij} = 1$
    \item Бóльшие размеры фильтра (5×5, 7×7) обеспечивают лучшую инвариантность, но требуют больше вычислений
    \item Наиболее эффективными показали себя биномиальные фильтры, соответствующие строкам треугольника Паскаля (например, [1,2,1], [1,4,6,4,1])
\end{itemize}

Экспериментальные исследования на различных архитектурах (AlexNet, VGG, ResNet, DenseNet) и датасетах (CIFAR-10, ImageNet) продемонстрировали, что BlurPool значительно повышает консистентность предсказаний при сдвигах входных данных без существенного ухудшения точности классификации. Например, на CIFAR-10 с архитектурой VGG13-bn применение биномиального фильтра размером 7×7 повысило консистентность с 88.1\% до 98.1\%, одновременно улучшив точность классификации с 91.6\% до 93.0\%.

Анализ внутренних представлений показал, что стандартные CNN ступенчато теряют эквивариантность после каждой операции субдискретизации, в то время как сети с BlurPool сохраняют периодическую эквивариантность меньших порядков. Дополнительно было выявлено, что BlurPool повышает устойчивость моделей к различным искажениям из тестовых наборов ImageNet-C/P, снижая среднюю ошибку классификации на эти искажения примерно на 3 процентных пункта.

Важным преимуществом метода является его простота и вычислительная эффективность: BlurPool требует менее 1\% дополнительных вычислений по сравнению с исходными моделями и может быть легко интегрирован в существующие архитектуры без необходимости полного переобучения. Это делает его практичным решением для широкого спектра приложений компьютерного зрения.

\subsection{Полифазные методы и их применение в CNN}
\label{review:antialias:tips_and_applications}

Более продвинутый подход к решению проблемы инвариантности предложен в работе Chaman and Dokmanić \cite{Chaman2021} — Translation Invariant Polyphase Sampling (TIPS). В отличие от BlurPool, который применяет фиксированный низкочастотный фильтр, TIPS использует полифазное разложение сигнала для явного моделирования и компенсации эффектов субдискретизации.

Метод TIPS опирается на теорию обработки сигналов и вводит количественную метрику для измерения проблемы — Maximum-Sampling Bias (MSB):

\begin{multline}
  \text{MSB} = \frac{1}{N}\sum_{n=1}^{N}\left(\max_{(i,j)\in\Omega_s}X_{n,i,j} - \mu_{\Omega_s}(X_n)\right),
\end{multline}

где $\Omega_s$ — окно с шагом $s$, а $\mu_{\Omega_s}(X_n)$ — среднее значение активаций $X_n$ по окну $\Omega_s$. Данная метрика измеряет склонность пулинга выбирать максимальные значения внутри окна, что экспериментально коррелирует с ухудшением инвариантности к сдвигам.

Основная идея TIPS заключается в том, что вместо прямой субдискретизации сигнала, которая вызывает потерю информации, сигнал разделяется на несколько «фаз» в соответствии с его позицией относительно сетки субдискретизации. Математически для входного тензора $X\in\mathbb{R}^{c\times h\times w}$ после ReLU-слоя, polyphase decomposition разбивает $X$ на $s^2$ взаимно дополняющих компонент:

\begin{equation}
  \text{poly}_{is+j}^{s}(X) = X[k,\,sn_1+i,\,sn_2+j], \quad \forall i,j\in\{0,\dots,s-1\},\;k,n_1,n_2,
\end{equation}

что эквивалентно свертке с шагом $s$ с ядром размера $1$ в позиции $(i,j)$.

Ключевое отличие TIPS от других методов — введение обучаемых смешивающих коэффициентов $\tau_{is+j}\in[0,1]$ для каждой компоненты. Эти коэффициенты генерируются небольшой shift-инвариантной нейронной сетью (последовательность $3\times3$ свёрток с ReLU), а итоговые активации вычисляются как взвешенная сумма:

\begin{equation}
  \hat{X} = \sum_{i,j} \tau_{is+j} \cdot \text{poly}_{is+j}^{s}(X).
\end{equation}

Для предотвращения вырожденных случаев, авторы вводят две специализированные регуляризации:

\begin{enumerate}
    \item Failure-Mode Penalty, которая предотвращает как скошенное (например, $\{0,1,0,0\}$), так и полностью равномерное распределение коэффициентов: $\mathcal{L}_{\text{FM}} = (1-s^2)\|\tau\|_2^2$.
    
    \item Undo-Shift Penalty, направленная на восстановление исходного сигнала при случайных сдвигах: $\mathcal{L}_{\text{undo}} = \|\text{TIPS}(X_{\Delta})-\text{TIPS}(X)\|_2^2$.
\end{enumerate}

Экспериментальные результаты показывают превосходство TIPS над существующими методами пулинга по метрикам консистентности и точности. На CIFAR-10 с ResNet-18 TIPS достигает 98.38\% консистентности при стандартных сдвигах по сравнению с 87.43\% у MaxPool и 91.04\% у BlurPool. При этом вычислительная сложность метода остается низкой: +3\% FLOP и менее 0.5\% дополнительных параметров по сравнению с MaxPool.

Абляционные исследования демонстрируют, что добавление TIPS только в слои со страйдом 2 улучшает консистентность на 3.4 процентных пункта, а установка во все уровни сети — на 6.1 п.п. Метод также повышает устойчивость к коррупциям и состязательным атакам, улучшая показатель mCE на ImageNet-C на 2.6 п.п. и снижая успех PGD-$\ell_\infty$ атак на 4 п.п.

Важным преимуществом TIPS является его архитектурная агностичность и простота интеграции в существующие нейросетевые пайплайны, включая как классификаторы, так и модели сегментации на основе UNet и DeepLabV3+.

\subsection{Адаптивный антиалиасинг}
\label{review:antialias:adaptive}

Zou и соавторы \cite{Zou2020} предложили более гибкий метод борьбы с алиасингом, получивший название Adaptive Anti-aliasing (AA). В отличие от BlurPool, где используется одинаковый фиксированный фильтр для всего изображения, адаптивный подход предполагает генерацию контент-зависимых низкочастотных фильтров.

Ключевая идея метода заключается в том, что разные области изображения требуют разной степени фильтрации: высокочастотные детали (края, текстуры) следует сохранять, в то время как шумные участки можно сильнее размывать. Для этого авторы обучают небольшую CNN, которая предсказывает специфические фильтры для:
\begin{itemize}
    \item каждой пространственной позиции $(i,j)$ в карте признаков;
    \item разных групп каналов, что позволяет дифференцированно обрабатывать различные типы признаков.
\end{itemize}

Математически для каждой позиции $(i,j)$ и группы каналов $g$ генерируется фильтр $w_{i,j,g} \in \mathbb{R}^{k\times k}$ (обычно $k=3$). Веса фильтра нормализуются с помощью softmax для обеспечения положительности и единичной суммы, что гарантирует свойство низкочастотного пропускания.

Эксперименты на ImageNet показали, что адаптивный антиалиасинг превосходит фиксированные методы как по точности классификации (+1.3 процентных пункта к Top-1), так и по консистентности предсказаний при сдвигах (+0.2-0.4 п.п.). Особенно заметным оказалось улучшение при переносе на другие домены, например, на видеоданные ImageNet VID (+2.3 п.п. к точности).

Применение метода к задачам сегментации также продемонстрировало существенные улучшения: в instance segmentation на MS-COCO наблюдался прирост в +1.1 п.п. по mAP и +4.1 п.п. по метрике консистентности масок (mAISC), а в semantic segmentation на наборах данных PASCAL VOC и Cityscapes — прирост в +1.0-1.8 п.п. по mIOU.

Визуальный анализ показал, что адаптивный антиалиасинг лучше сохраняет детали объектов, особенно на границах и текстурированных участках, при этом эффективно подавляя шум в однородных областях. Ресурсоёмкость метода остаётся умеренной — увеличение числа параметров составляет не более 7.8\% от базовой модели, а время инференса возрастает примерно в 1.5 раза.

\section{Специфические проблемы инвариантности в детекторах объектов}
\label{review:detectors}

Детекция объектов представляет собой более сложную задачу по сравнению с классификацией изображений, поскольку требует не только определения класса объекта, но и точной локализации его положения на изображении. Это делает проблему инвариантности к сдвигам особенно критичной для детекторов объектов.

\subsection{Архитектуры современных детекторов и влияние алиасинга}
\label{review:detectors:architectures}

Современные детекторы объектов широко используют CNN в качестве основы для извлечения признаков и наследуют проблемы инвариантности к сдвигам, присущие этим архитектурам. Однако из-за необходимости точной локализации объектов, эти проблемы проявляются более ярко.

Moskvyak et al. \cite{Moskvyak2021} показали, что проблема особенно выражена в одностадийных детекторах, таких как YOLO. Это связано с тем, что эти детекторы используют фиксированную сетку для предсказания ограничивающих рамок, и небольшие изменения в представлениях признаков могут привести к тому, что объект будет ассоциирован с другой ячейкой сетки, вызывая значительное изменение в предсказании.

\subsection{Метрики устойчивости и адаптация методов анти-алиасинга}
\label{review:detectors:adaptation}

Для оценки устойчивости детекторов используются специфические метрики, такие как стабильность IoU, дрейф центра ограничивающей рамки и стабильность уверенности.

Lin et al. \cite{Lin2020} предложили подход к интеграции BlurPool в архитектуру Faster R-CNN. Для одностадийных детекторов, таких как YOLO, Wang et al. \cite{Wang2020} предложили специализированную версию BlurPool, учитывающую особенности архитектуры с множественными выходами на разных масштабах.

Более сложный подход, основанный на TIPS, был адаптирован для детекторов объектов в работе Chaman et al. \cite{Chaman2022}. Авторы модифицировали архитектуру YOLOv3, заменив стандартные операции даунсэмплинга на TIPS-модули, и показали, что это приводит к значительному улучшению стабильности предсказаний.

Решение проблемы инвариантности к сдвигам в детекторах объектов является не только теоретически интересной задачей, но и имеет важное практическое значение для повышения надежности и безопасности систем компьютерного зрения в критически важных приложениях. 