\chapter{Обзор литературы} \label{review}

\section{Инвариантность к сдвигу в CNN-классификаторах}
\label{review:invariance}

Сверточные нейронные сети в теории должны обладать определенной степенью инвариантности к позиционным сдвигам входных данных. Это свойство изначально заложено в их архитектуру через механизм разделения весов и локальные рецептивные поля \cite{LeCun1998}. Однако, как показывают многочисленные исследования последних лет, современные CNN демонстрируют ограниченную инвариантность к сдвигам, что противоречит интуитивным ожиданиям.

\subsection{Теоретические основы инвариантности в CNN}
\label{review:invariance:theory}

Одной из первых работ, в которой было формально описано свойство эквивариантности свёрточных сетей к сдвигам, является исследование LeCun et al. \cite{LeCun1998}. В этой работе авторы выделили ключевые свойства CNN — локальность связей, разделение весов и пространственный пулинг — которые в комбинации должны обеспечивать устойчивость к пространственным искажениям входных данных. В частности, авторы указывали, что операция свёртки сама по себе обладает эквивариантностью к сдвигам, то есть если входное изображение сдвигается, то соответствующим образом сдвигаются и карты признаков, формируемые свёрточными слоями.

Дальнейшее теоретическое развитие эта идея получила в работах Mallat \cite{Mallat2012}, где была предложена теория рассеяния (scattering theory), обосновывающая математические принципы построения инвариантных к различным преобразованиям представлений сигналов. В контексте сверточных сетей эта теория дает формальную основу для понимания того, как многослойные архитектуры способны формировать признаки, устойчивые к различным искажениям, включая сдвиги.

Однако теоретические предпосылки часто расходятся с практикой. Simoncelli et al. \cite{Simoncelli1995} еще в 1995 году указывали на проблему алиасинга при субдискретизации сигналов, которая впоследствии была идентифицирована как одна из ключевых причин нарушения инвариантности к сдвигам в CNN. В традиционной обработке сигналов перед снижением частоты дискретизации применяется низкочастотная фильтрация для предотвращения алиасинга, но в стандартных CNN эта практика долгое время игнорировалась.

\subsection{Эмпирические исследования проблемы}
\label{review:invariance:empirical}

Несмотря на теоретические ожидания, ряд эмпирических исследований показал ограниченную инвариантность современных CNN к сдвигам. Одной из первых фундаментальных работ в этом направлении стало исследование Engstrom et al. \cite{Engstrom2019}, в котором авторы продемонстрировали, что даже небольшие сдвиги или повороты входных изображений могут значительно снизить точность классификации современных CNN, включая ResNet и другие state-of-the-art архитектуры.

Zhang \cite{Zhang2019} провел более детальное исследование проблемы и идентифицировал операции даунсэмплинга (в частности, max-pooling и свертку с шагом больше 1) как основной источник нарушения инвариантности к сдвигам. В этой работе было показано, что субпиксельные сдвиги входных изображений могут приводить к значительным изменениям в активациях нейронов и, как следствие, к нестабильности выходных предсказаний модели.

Azulay and Weiss \cite{Azulay2018} пошли дальше и продемонстрировали, что проблема инвариантности в CNN может быть систематически исследована через призму классической теории обработки сигналов. Они показали, что отсутствие антиалиасинговых фильтров перед операциями субдискретизации приводит к высокочастотному шуму в представлениях признаков, что делает модель чувствительной к малым сдвигам входных данных.

Chaman и Dokmanic \cite{Chaman2021} более формально исследовали эффекты алиасинга в CNN и предложили метрики для количественной оценки степени инвариантности моделей к различным преобразованиям. Их исследование также подтвердило, что стандартные архитектуры CNN, такие как VGG и ResNet, демонстрируют ограниченную инвариантность к сдвигам, особенно при наличии субпиксельных смещений.

Работа Kayhan и van Gemert \cite{Kayhan2020} обратила внимание еще на один интересный аспект проблемы — они показали, что современные CNN в процессе обучения могут "запоминать" абсолютные позиции объектов в кадре из обучающей выборки, что также противоречит желаемому свойству инвариантности к положению. Это явление они назвали "позиционным кодированием" (position encoding) и предложили методы его смягчения.

\subsection{Количественные метрики инвариантности}
\label{review:invariance:metrics}

Для объективного сравнения степени инвариантности различных архитектур CNN к сдвигам необходимы формальные метрики. Одним из распространенных подходов является измерение косинусного сходства между векторами признаков, полученными из оригинального и сдвинутого изображений.

Zhang \cite{Zhang2019} предложил метрику стабильности предсказаний, основанную на среднем изменении выходных вероятностей модели при субпиксельных сдвигах входных данных. Эта метрика позволяет количественно оценить, насколько стабильны решения модели при малых пространственных возмущениях входа.

Более сложные метрики были предложены в работе Chaman и Dokmanic \cite{Chaman2021}, где авторы ввели понятие "translation discrepancy function" (TDF), которая измеряет максимальное изменение в выходе модели при всех возможных сдвигах входного изображения в определенном диапазоне.

В контексте задач детекции объектов Manfredi и Wang \cite{Manfredi2020} предложили использовать стабильность IoU (Intersection over Union) и дрейф центра ограничивающей рамки как метрики инвариантности к сдвигам. Эти метрики позволяют оценить, насколько стабильно модель локализует объекты при малых сдвигах входных изображений.

\subsection{Влияние архитектурных особенностей на инвариантность}
\label{review:invariance:architectures}

Различные архитектуры CNN демонстрируют разную степень инвариантности к сдвигам, что обусловлено их структурными особенностями. Исследования показывают, что более глубокие сети, такие как ResNet \cite{He2016}, как правило, более инвариантны к сдвигам по сравнению с менее глубокими архитектурами, такими как AlexNet или VGG \cite{Simonyan2014}.

Sabour et al. \cite{Sabour2017} в своей работе по капсульным сетям указывали на фундаментальные ограничения CNN в части обеспечения инвариантности к сдвигам и предложили альтернативную архитектуру, в которой явно моделируются пространственные отношения между частями объектов.

Ряд исследований также показал влияние типа пулинга на инвариантность к сдвигам. В частности, работа Scherer et al. \cite{Scherer2010} сравнивала различные типы пулинга (max, average, stochastic) и их влияние на обобщающую способность и инвариантность моделей. Zhang \cite{Zhang2019} позже показал, что average-pooling обеспечивает лучшую инвариантность к сдвигам по сравнению с max-pooling, хотя может уступать в общей точности классификации.

Исследование Blot et al. \cite{Blot2016} обратило внимание на влияние размера рецептивного поля на инвариантность к сдвигам. Авторы продемонстрировали, что модели с большими рецептивными полями, как правило, более устойчивы к пространственным преобразованиям входных данных.

Таким образом, обзор литературы по инвариантности к сдвигам в CNN-классификаторах показывает, что эта проблема имеет глубокие теоретические основы, подтверждается многочисленными эмпирическими исследованиями и зависит от множества архитектурных факторов. Для ее решения необходимы как теоретически обоснованные подходы, так и практические методы, учитывающие специфику современных архитектур CNN.

\section{Методы анти-алиасинга в нейронных сетях}
\label{review:antialias}

После идентификации алиасинга как основной причины нарушения инвариантности к сдвигам в CNN, исследователи предложили ряд методов для решения этой проблемы, основанных на принципах классической обработки сигналов и адаптированных к особенностям нейронных сетей.

\subsection{Низкочастотная фильтрация и BlurPool}
\label{review:antialias:blurpool}

Наиболее прямолинейным подходом к борьбе с алиасингом является применение низкочастотной фильтрации перед операциями субдискретизации, что соответствует классической теории обработки сигналов. Этот подход был впервые систематически применен к CNN в работе Zhang \cite{Zhang2019}, где был предложен метод BlurPool (Blur-then-downsampling).

В BlurPool операции max-pooling и свертки с шагом больше 1 модифицируются таким образом, что перед непосредственной субдискретизацией применяется размытие с использованием фиксированного низкочастотного фильтра. Авторы исследовали различные типы фильтров, включая простое усреднение (box filter), треугольный фильтр (binomial filter) и фильтр Гаусса, и показали, что даже простейшие из них значительно улучшают инвариантность сети к сдвигам.

Важным преимуществом BlurPool является его архитектурная простота и возможность интеграции в существующие модели без необходимости переобучения с нуля. Замена стандартных операций пулинга и свертки с шагом на их «размытые» аналоги может быть выполнена постфактум в предобученных моделях с сохранением большей части их весов.

Последующие исследования показали эффективность BlurPool для различных архитектур CNN. Например, Zou et al. \cite{Zou2020} продемонстрировали, что применение BlurPool к архитектурам ResNet не только улучшает их инвариантность к сдвигам, но и повышает устойчивость к состязательным атакам (adversarial attacks).

\subsection{Полифазная выборка с инвариантностью к сдвигам (TIPS)}
\label{review:antialias:tips}

Альтернативный и более сложный подход к обеспечению инвариантности к сдвигам был предложен в работе Chaman и Dokmanic \cite{Chaman2021} под названием Translation Invariant Polyphase Sampling (TIPS). В отличие от BlurPool, который применяет фиксированный низкочастотный фильтр, TIPS использует полифазное разложение сигнала для явного моделирования и компенсации эффектов субдискретизации.

Основная идея TIPS заключается в том, что вместо прямой субдискретизации сигнала, вызывающей потерю информации, сигнал разделяется на несколько «фаз» в соответствии с его позицией относительно сетки субдискретизации. Затем каждая фаза обрабатывается отдельно, после чего результаты объединяются таким образом, чтобы получить представление, инвариантное к исходному положению сигнала.

Математически TIPS можно рассматривать как обобщение идеи кросс-корреляции с циклическим сдвигом, которая гарантирует, что выход модели будет одинаковым для всех целочисленных сдвигов входного сигнала. TIPS распространяет этот принцип на субпиксельные (нецелочисленные) сдвиги, обеспечивая более полную инвариантность.

Исследования показывают, что TIPS обеспечивает наилучшую теоретическую гарантию инвариантности к сдвигам среди существующих методов, хотя и требует более значительных изменений в архитектуре сети и может быть вычислительно более затратным по сравнению с BlurPool.

\subsection{Другие подходы к обеспечению инвариантности}
\label{review:antialias:other}

Помимо BlurPool и TIPS, в литературе предложены и другие подходы к улучшению инвариантности CNN к сдвигам.

Одним из таких подходов является Deep Shift-Invariant Network (DSI), предложенный Zou et al. \cite{Zou2021}. DSI основан на идее моделирования сдвига как дифференцируемой операции и использования глубокого обучения для адаптации параметров размытия к конкретной задаче. В отличие от BlurPool, где параметры фильтра фиксированы, в DSI они оптимизируются в процессе обучения модели.

Другой подход, предложенный в работе Anwar et al. \cite{Anwar2020}, использует идею рандомизированного пулинга (Random Sampling Pooling), где позиции для субдискретизации выбираются случайным образом во время обучения. Это помогает модели не «привязываться» к фиксированным позициям при субдискретизации и лучше генерализоваться на сдвинутые входные данные.

Интересным направлением является также использование явной аугментации данных для улучшения инвариантности. Например, Cheng et al. \cite{Cheng2019} предложили метод Shift Equivariance Regularization (SER), который во время обучения явно стимулирует модель давать согласованные предсказания для оригинальных и сдвинутых версий входных изображений.

\subsection{Применение анти-алиасинга в различных задачах}
\label{review:antialias:applications}

Методы анти-алиасинга нашли применение в различных задачах компьютерного зрения, выходящих за рамки простой классификации изображений.

В области сегментации изображений Zheng et al. \cite{Zheng2021} показали, что применение BlurPool к архитектурам U-Net и DeepLab значительно улучшает стабильность границ сегментации при малых сдвигах входных изображений, что особенно важно в медицинских приложениях.

В контексте генеративных моделей Karras et al. \cite{Karras2021} продемонстрировали, что включение анти-алиасинговых фильтров в генеративно-состязательные сети (GAN) помогает устранить характерные артефакты в генерируемых изображениях и улучшает стабильность процесса обучения.

Для задач детекции объектов Lin et al. \cite{Lin2020} адаптировали методы анти-алиасинга к популярным архитектурам детекторов, таким как Faster R-CNN и YOLO, и показали, что это значительно улучшает стабильность предсказаний ограничивающих рамок при малых сдвигах объектов.

В целом, обзор литературы по методам анти-алиасинга в нейронных сетях показывает, что применение принципов классической обработки сигналов к современным архитектурам CNN является эффективным подходом к улучшению их инвариантности к сдвигам. Различные методы, от простого BlurPool до более сложных подходов, основанных на полифазной выборке или адаптивных фильтрах, предоставляют широкий спектр инструментов для повышения устойчивости моделей к пространственным преобразованиям входных данных с различными компромиссами между вычислительной сложностью, архитектурными изменениями и степенью достигаемой инвариантности.

\section{Специфические проблемы инвариантности в детекторах объектов}
\label{review:detectors}

Детекция объектов представляет собой более сложную задачу по сравнению с классификацией изображений, поскольку требует не только определения класса объекта, но и точной локализации его положения на изображении. Это делает проблему инвариантности к сдвигам особенно критичной для детекторов объектов, так как даже небольшие нарушения стабильности могут привести к значительным ошибкам в определении положения и размеров ограничивающих рамок.

\subsection{Архитектуры современных детекторов объектов}
\label{review:detectors:architectures}

Современные детекторы объектов можно разделить на две основные категории: двухстадийные и одностадийные.

Двухстадийные детекторы, такие как R-CNN \cite{Girshick2014} и его последователи (Fast R-CNN \cite{Girshick2015}, Faster R-CNN \cite{Ren2015}), сначала генерируют набор потенциальных областей интереса (region proposals), а затем классифицируют эти области и уточняют их координаты. Такой подход обеспечивает высокую точность, но может иметь ограничения по скорости работы.

Одностадийные детекторы, такие как YOLO \cite{Redmon2016} и SSD \cite{Liu2016}, выполняют определение класса и локализацию объектов напрямую, без промежуточного этапа генерации предложений. Это позволяет им работать значительно быстрее, что критично для приложений реального времени, хотя исторически они уступали двухстадийным детекторам по точности.

Обе категории детекторов широко используют CNN в качестве основы для извлечения признаков, и поэтому наследуют проблемы инвариантности к сдвигам, присущие этим архитектурам. Однако, из-за необходимости точной локализации объектов, эти проблемы проявляются в детекторах более ярко и имеют специфические аспекты.

\subsection{Влияние алиасинга на стабильность детекции}
\label{review:detectors:aliasing}

Исследования показывают, что алиасинг и связанная с ним нестабильность представлений в CNN имеют особенно серьезные последствия для задач детекции объектов. В работе Manfredi и Wang \cite{Manfredi2020} авторы продемонстрировали, что небольшие субпиксельные сдвиги входных изображений могут приводить к значительным изменениям в предсказанных ограничивающих рамках даже для современных детекторов.

Одной из ключевых проблем является дрейф центра ограничивающей рамки — явление, при котором центр предсказанной рамки смещается при изменении положения объекта на изображении. Это особенно критично для задач, требующих высокой точности локализации, таких как медицинская диагностика или прецизионная робототехника.

Moskvyak et al. \cite{Moskvyak2021} исследовали влияние различных архитектурных компонентов детекторов на их устойчивость к сдвигам и показали, что проблема особенно выражена в одностадийных детекторах, таких как YOLO. Это связано с тем, что эти детекторы используют фиксированную сетку для предсказания ограничивающих рамок, и небольшие изменения в представлениях признаков могут привести к тому, что объект будет ассоциирован с другой ячейкой сетки, вызывая значительное изменение в предсказании.

Авторы также отметили, что проблема усугубляется для объектов малого размера и объектов, расположенных на границах ячеек предсказания, что делает детекторы особенно уязвимыми к сдвигам в реальных сценариях, где положение объектов не контролируется.

\subsection{Метрики устойчивости детекторов}
\label{review:detectors:metrics}

Для оценки устойчивости детекторов объектов к пространственным преобразованиям входных данных используются специфические метрики, отражающие стабильность как классификационных, так и локализационных аспектов задачи.

Одной из ключевых метрик является стабильность IoU (Intersection over Union), которая измеряет, насколько сильно изменяется перекрытие между предсказанной и истинной ограничивающими рамками при сдвиге входного изображения. Низкая стабильность IoU указывает на чувствительность детектора к малым пространственным преобразованиям входа.

Другой важной метрикой является дрейф центра ограничивающей рамки, который измеряет среднее смещение центра предсказанной рамки при сдвиге входного изображения. Эта метрика особенно важна для оценки точности локализации объектов и может быть измерена как в абсолютных (пиксели), так и в относительных единицах (в процентах от размера объекта).

Стабильность уверенности детекции (confidence stability) измеряет, насколько стабильны значения уверенности модели в своих предсказаниях при малых сдвигах входа. Высокая вариация уверенности может приводить к проблемам с пороговой фильтрацией в реальных приложениях.

Yang et al. \cite{Yang2022} предложили комплексную метрику инвариантности детекторов, которая объединяет все эти аспекты и позволяет количественно сравнивать различные архитектуры и методы по их устойчивости к пространственным преобразованиям.

\subsection{Адаптация методов анти-алиасинга для детекторов}
\label{review:detectors:adaptation}

Адаптация методов анти-алиасинга, разработанных для классификационных моделей, к детекторам объектов представляет собой нетривиальную задачу из-за сложности архитектур детекторов и специфики задачи локализации.

Lin et al. \cite{Lin2020} предложили подход к интеграции BlurPool в архитектуру Faster R-CNN, модифицируя как базовую сеть извлечения признаков, так и модуль предложения регионов (Region Proposal Network, RPN). Авторы показали, что такая модификация значительно улучшает стабильность предсказаний ограничивающих рамок при малых сдвигах входных изображений без существенного влияния на общую точность детекции.

Для одностадийных детекторов, таких как YOLO, Wang et al. \cite{Wang2020} предложили специализированную версию BlurPool, которая учитывает особенности архитектуры с множественными выходами на разных масштабах. Их подход заключается во внедрении анти-алиасинговых фильтров на каждом уровне пирамиды признаков, что позволяет улучшить инвариантность к сдвигам для объектов разного размера.

Более сложный подход, основанный на TIPS, был адаптирован для детекторов объектов в работе Chaman et al. \cite{Chaman2022}. Авторы модифицировали архитектуру YOLOv3, заменив стандартные операции даунсэмплинга на TIPS-модули, и показали, что это приводит к значительному улучшению стабильности предсказаний, особенно для объектов малого размера.

\subsection{Практические последствия нестабильности детекторов}
\label{review:detectors:implications}

Нестабильность детекторов объектов при малых сдвигах входных данных имеет серьезные практические последствия в различных приложениях.

В системах видеонаблюдения и отслеживания объектов нестабильность может приводить к прерывистым траекториям и ложным срабатываниям алгоритмов трекинга, особенно при наличии вибраций камеры или других источников малых сдвигов в последовательности кадров.

В беспилотных транспортных средствах и роботах нестабильность детекции может влиять на точность определения положения препятствий и других участников движения, что критично для безопасности. Даже небольшие ошибки в предсказании расстояния до объекта могут привести к неправильным решениям системы управления.

В медицинских приложениях, таких как автоматический анализ рентгеновских снимков или МРТ, нестабильность может привести к неточной локализации патологий или ложным срабатываниям, что может повлиять на диагностические решения.

Решение проблемы инвариантности к сдвигам в детекторах объектов является, таким образом, не только теоретически интересной задачей, но и имеет важное практическое значение для повышения надежности и безопасности систем компьютерного зрения в критически важных приложениях.

В целом, обзор литературы показывает, что проблема инвариантности к сдвигам представляет особый интерес и сложность в контексте детекторов объектов. Современные подходы к ее решению, такие как BlurPool и TIPS, демонстрируют обнадеживающие результаты, но требуют специфической адаптации к архитектурам детекторов и особенностям задачи локализации объектов.

\section{Другие типы инвариантности в нейронных сетях}
\label{review:other_invariance}

Хотя проблема инвариантности к сдвигам является одной из наиболее изученных в контексте CNN, существуют и другие типы инвариантности, которые также важны для построения надежных систем компьютерного зрения. Эти типы инвариантности отражают различные преобразования, которым могут подвергаться объекты в реальном мире, такие как изменение масштаба, поворот, аффинные преобразования и другие.

\subsection{Инвариантность к масштабу}
\label{review:other_invariance:scale}

Инвариантность к масштабу — это способность модели одинаково эффективно распознавать объекты независимо от их размера в кадре. Эта проблема особенно актуальна для приложений, где расстояние до объектов может значительно меняться.

Стандартным подходом к обеспечению инвариантности к масштабу в современных CNN является использование многоуровневых (многомасштабных) представлений. Архитектуры, такие как Feature Pyramid Network (FPN) \cite{Lin2017}, интегрируют информацию с разных уровней сети, создавая пирамиду признаков с разным разрешением и рецептивным полем.

Альтернативный подход — это явное моделирование преобразований масштаба. Например, в работе Kanazawa et al. \cite{Kanazawa2014} был предложен метод под названием "Locally Scale-Invariant Convolutional Neural Networks", который модифицирует свёрточные слои таким образом, чтобы они становились инвариантными к локальным изменениям масштаба.

Xu et al. \cite{Xu2014} предложили Scale-Invariant Convolutional Neural Network (SiCNN), в которой входное изображение обрабатывается на нескольких масштабах параллельно, после чего результаты объединяются. Этот подход позволяет явно моделировать инвариантность к масштабу на уровне архитектуры.

В контексте детекции объектов проблема инвариантности к масштабу особенно важна, поскольку объекты могут появляться в широком диапазоне размеров. Современные архитектуры детекторов, такие как RetinaNet \cite{Lin2017retinanet} и YOLOv3 \cite{Redmon2018}, используют пирамиды признаков для обнаружения объектов разного размера.

\subsection{Инвариантность к повороту}
\label{review:other_invariance:rotation}

Инвариантность к повороту — это способность модели одинаково распознавать объекты независимо от их ориентации в пространстве. Эта проблема важна для многих приложений, где ориентация объектов может быть произвольной, например, в спутниковых снимках, медицинских изображениях и распознавании текстур.

Классические CNN не обладают встроенной инвариантностью к повороту, и для решения этой проблемы были предложены различные подходы. Один из них — аугментация данных через случайные повороты во время обучения. Хотя этот подход прост и эффективен, он может требовать значительного увеличения размера обучающей выборки и времени обучения.

Более элегантные архитектурные решения включают Rotation Equivariant Vector Field Networks (RotEqNet) \cite{Marcos2017}, которые используют фильтры, способные явно моделировать вращения входных данных, и Harmonic Networks \cite{Worrall2017}, которые основаны на использовании гармонических фильтров, инвариантных к вращению.

Cohen и Welling \cite{Cohen2016} предложили Group Equivariant Convolutional Networks (G-CNNs), обобщение CNN, которое обеспечивает эквивариантность к более широкому классу преобразований, включая повороты, отражения и другие действия определенных групп симметрий.

Weiler et al. \cite{Weiler2018} развили эти идеи дальше, предложив Steerable CNNs, которые используют теорию представлений групп для построения фильтров, естественным образом моделирующих вращательные симметрии.

В области детекции объектов инвариантность к повороту особенно важна для задач, где ориентация объектов может быть произвольной, таких как аэрофотосъемка или сканирование багажа в аэропортах. Zhou et al. \cite{Zhou2017} предложили Oriented Response Networks (ORN), которые расширяют стандартные CNN для обработки входных данных в различных ориентациях и показывают хорошие результаты в задачах детекции и сегментации объектов произвольной ориентации.

\subsection{Другие типы геометрических инвариантностей}
\label{review:other_invariance:others}

Помимо сдвигов, масштаба и поворотов, существуют и другие типы геометрических преобразований, к которым может быть полезно обеспечить инвариантность моделей.

Аффинные преобразования, которые включают в себя линейные деформации (сжатие, растяжение, сдвиг), часто встречаются в реальных сценариях из-за перспективных искажений и различных условий съемки. Jaderberg et al. \cite{Jaderberg2015} предложили архитектуру Spatial Transformer Networks (STN), которая способна обучаться локализовать и выравнивать релевантные части входного изображения перед его дальнейшей обработкой, что позволяет достичь инвариантности к широкому классу пространственных преобразований.

Проективные преобразования, возникающие из-за перспективы, также могут значительно влиять на внешний вид объектов. Esteves et al. \cite{Esteves2018} предложили подход, основанный на сферических представлениях, который обеспечивает инвариантность к более широкому классу 3D-преобразований.

Инвариантность к деформациям, таким как изгибы и складки, важна для распознавания эластичных объектов, таких как одежда или биологические структуры. В работе Sun et al. \cite{Sun2016} была предложена архитектура Deformation Convolutional Network (DCN), которая адаптирует форму рецептивных полей для моделирования деформаций и обеспечивает лучшую инвариантность к этому типу преобразований.

\subsection{Связь между различными типами инвариантности}
\label{review:other_invariance:connection}

Интересным аспектом исследования является связь между различными типами инвариантности и возможность их совместного обеспечения в рамках одной архитектуры.

Lenc и Vedaldi \cite{Lenc2019} исследовали теоретическую связь между различными типами инвариантности и показали, что некоторые из них могут быть несовместимы или требовать компромиссов. Например, стремление к полной инвариантности к повороту может негативно влиять на способность модели различать некоторые классы объектов (например, '6' и '9').

Другой важный аспект — это различие между инвариантностью и эквивариантностью. В то время как инвариантность означает, что выход модели не меняется при преобразовании входа, эквивариантность означает, что выход модели преобразуется предсказуемым образом при преобразовании входа. В некоторых задачах, особенно связанных с локализацией, требуется именно эквивариантность, а не полная инвариантность.

Современные подходы, такие как Group Equivariant CNNs и Steerable CNNs, предлагают общую теоретическую основу для моделирования различных типов геометрических инвариантностей и эквивариантностей в рамках единой архитектуры. Эти подходы позволяют достичь баланса между различными типами инвариантности в зависимости от требований конкретной задачи.

\section{Обучение с функцией потерь согласованности}
\label{review:consistency}

Помимо архитектурных модификаций для обеспечения инвариантности к различным преобразованиям, важным направлением исследований является разработка специальных функций потерь, которые явно стимулируют модель давать согласованные результаты при преобразованиях входных данных. Этот подход, известный как обучение с функцией потерь согласованности (consistency loss training), представляет собой мощную альтернативу или дополнение к архитектурным модификациям.

\subsection{Основные принципы обучения с согласованностью}
\label{review:consistency:principles}

Центральная идея обучения с согласованностью заключается в том, чтобы явно включить в функцию потерь требование, что выходы модели для оригинального и преобразованного входов должны быть связаны определенным образом, соответствующим примененному преобразованию.

Формально, если $f$ — это модель, $x$ — входное изображение, а $T$ — преобразование (например, сдвиг, поворот или изменение масштаба), то можно определить потери согласованности как:

\begin{equation}
L_{consistency} = d(f(x), g(f(T(x))))
\end{equation}

где $d$ — это некоторая мера расстояния (например, среднеквадратичная ошибка или дивергенция Кульбака-Лейблера), а $g$ — это функция, которая преобразует выход модели для трансформированного входа таким образом, чтобы он был сопоставим с выходом для оригинального входа.

Cheng et al. \cite{Cheng2019} предложили метод Shift Equivariance Regularization (SER), который добавляет к стандартной функции потерь классификации дополнительный член, штрафующий модель за несогласованность предсказаний при сдвигах входного изображения. Авторы показали, что такой подход может значительно улучшить инвариантность CNN к сдвигам без необходимости внесения изменений в их архитектуру.

Подобный подход был расширен на другие типы преобразований в работе Zhang et al. \cite{Zhang2020}, где авторы предложили общую структуру для обучения с согласованностью для различных геометрических преобразований.

\subsection{Полуавтоматическое и самоконтролируемое обучение}
\label{review:consistency:semi_self}

Важным приложением обучения с согласованностью является полуавтоматическое и самоконтролируемое обучение, где ограниченное количество размеченных данных дополняется большим количеством неразмеченных данных.

Xie et al. \cite{Xie2020} предложили метод Unsupervised Data Augmentation (UDA), который использует согласованность предсказаний между оригинальными и аугментированными версиями неразмеченных изображений как сигнал для обучения. Этот подход позволяет эффективно использовать неразмеченные данные для улучшения обобщающей способности модели.

Sohn et al. \cite{Sohn2020} развили эту идею в методе FixMatch, который сочетает согласованность предсказаний для слабо и сильно аугментированных версий неразмеченных изображений с псевдомаркировкой. Этот подход достиг впечатляющих результатов в задачах полуавтоматического обучения с очень ограниченным количеством размеченных данных.

\subsection{Применение в детекции объектов}
\label{review:consistency:detection}

Обучение с согласованностью нашло применение и в задачах детекции объектов, где стабильность предсказаний при преобразованиях входных данных особенно важна.

Wang et al. \cite{Wang2021} предложили метод Consistency-based Semi-supervised Object Detection, который использует согласованность предсказаний ограничивающих рамок между различными аугментированными версиями неразмеченных изображений для улучшения обучения детектора объектов.

Zhou et al. \cite{Zhou2021} развили эту идею дальше, предложив подход Spatial-Temporal Consistency for Semi-supervised Object Detection in Videos, который использует не только пространственную согласованность в рамках одного кадра, но и временную согласованность между последовательными кадрами видео.

\subsection{Преимущества и ограничения}
\label{review:consistency:advantages}

Обучение с согласованностью имеет ряд преимуществ по сравнению с чисто архитектурными подходами к обеспечению инвариантности:

1. Гибкость: Может быть применено к широкому спектру архитектур без необходимости их модификации.
2. Адаптивность: Модель может обучиться выборочной инвариантности, которая важна для конкретной задачи, вместо универсальной инвариантности, которая может быть избыточной или даже вредной.
3. Эффективность: Не требует увеличения числа параметров или вычислительной сложности модели во время инференса.

Однако есть и определенные ограничения:

1. Сложность обучения: Дополнительные члены в функции потерь могут усложнить процесс оптимизации и требовать тщательной настройки гиперпараметров.
2. Вычислительные затраты при обучении: Требуется вычислять выходы модели для нескольких версий каждого изображения, что может значительно увеличить время обучения.
3. Неполная гарантия: В отличие от некоторых архитектурных подходов, нет теоретической гарантии полной инвариантности, а лишь эмпирическое улучшение.

Тем не менее, обучение с согласованностью представляет собой мощный и гибкий инструмент для улучшения инвариантности моделей к различным преобразованиям и является важным дополнением к архитектурным подходам, рассмотренным ранее.

\section{Выводы по обзору литературы}
\label{review:conclusions}

Проведенный обзор литературы позволяет сделать следующие выводы относительно проблемы пространственной инвариантности в сверточных нейронных сетях:

\begin{enumerate}
    \item Проблема отсутствия полной инвариантности к сдвигам в современных CNN хорошо задокументирована и является существенным ограничением для многих практических приложений. Несмотря на теоретические предпосылки к эквивариантности операции свертки, использование операций субдискретизации без должной фильтрации приводит к алиасингу и нарушению этого свойства.
    
    \item Основной причиной нарушения инвариантности к сдвигам является алиасинг, возникающий при операциях даунсэмплинга, таких как max-pooling и свертка с шагом больше 1. Этот эффект хорошо изучен в классической теории обработки сигналов, но долгое время игнорировался в дизайне CNN.
    
    \item Для решения проблемы предложены различные методы, основанные на принципах обработки сигналов, включая BlurPool (низкочастотную фильтрацию перед даунсэмплингом) и TIPS (полифазную выборку с инвариантностью к сдвигам). Эти методы показывают значительное улучшение инвариантности CNN к сдвигам при относительно небольших изменениях в архитектуре.
    
    \item Проблема инвариантности особенно критична для детекторов объектов, где нестабильность предсказаний ограничивающих рамок может иметь серьезные последствия в приложениях реального времени. Одностадийные детекторы, такие как YOLO, особенно подвержены этой проблеме из-за использования фиксированной сетки предсказаний.
    
    \item Помимо инвариантности к сдвигам, важны и другие типы инвариантности, такие как инвариантность к масштабу и повороту. Для их обеспечения также разработаны специализированные архитектурные решения, такие как пирамиды признаков и группо-эквивариантные свертки.
    
    \item Альтернативным подходом к архитектурным модификациям является обучение с функцией потерь согласованности, которое явно стимулирует модель давать согласованные предсказания при преобразованиях входных данных. Этот подход особенно полезен в контексте полуавтоматического и самоконтролируемого обучения.
    
    \item Большинство исследований в области инвариантности к сдвигам фокусируется на классификационных задачах, и относительно мало работ посвящено систематическому изучению этой проблемы в контексте детекции объектов, особенно с использованием современных методов, таких как TIPS.
\end{enumerate}

Таким образом, несмотря на значительный прогресс в понимании и решении проблемы пространственной инвариантности в CNN, остаются открытые вопросы, особенно в контексте детекторов объектов и сложных реальных сценариев. Систематическое исследование влияния методов повышения инвариантности на различные аспекты работы детекторов, таких как YOLOv5, представляет собой важное и актуальное направление исследований, которое может привести к значительному улучшению стабильности и надежности систем компьютерного зрения в критически важных приложениях.
