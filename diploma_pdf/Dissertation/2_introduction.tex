\chapter*{Введение}							% Заголовок
\addcontentsline{toc}{chapter}{Введение}	% Добавляем его в оглавление
\label{intro}

\section*{Актуальность проблемы}
\label{intro:relevance}

Сверточные нейронные сети (CNN) сегодня являются ключевым инструментом в решении широкого спектра задач компьютерного зрения, включая классификацию изображений, сегментацию, детекцию объектов и другие. Их популярность и эффективность обусловлены способностью к автоматическому извлечению иерархии признаков из необработанных данных и высокой точностью работы в различных условиях. Теоретические основы CNN предполагают, что они должны обладать свойством инвариантности к пространственным преобразованиям, в частности, к сдвигам входных данных. Это означает, что одинаковые объекты, расположенные в разных частях изображения, должны распознаваться с одинаковой точностью и уверенностью.

Однако практика показывает, что современные архитектуры CNN не обладают полной инвариантностью к сдвигам. Небольшие, даже субпиксельные смещения объектов на входном изображении могут приводить к значительным изменениям в выходных результатах сети. Эта проблема, часто упускаемая из виду при традиционной оценке моделей на тестовых выборках, может иметь серьезные последствия в реальных приложениях компьютерного зрения, особенно в критически важных областях, таких как автономные транспортные средства, системы видеонаблюдения, медицинская диагностика и робототехника.

Отсутствие стабильности предсказаний при малых смещениях объектов может привести к:
\begin{itemize}
    \item Ложным срабатываниям или пропускам в системах обнаружения объектов
    \item Нестабильной работе алгоритмов слежения за объектами
    \item Некорректной сегментации медицинских изображений
    \item Ошибкам в системах управления роботами и беспилотными автомобилями
    \item Снижению надежности систем биометрической идентификации
\end{itemize}

Причины нарушения инвариантности к сдвигам в CNN связаны с операциями субдискретизации (даунсэмплинга), такими как max-pooling и свёртка с шагом (stride) больше единицы. Эти операции позволяют уменьшать пространственное разрешение карт признаков, что необходимо для снижения вычислительной сложности и обобщающей способности сети, но одновременно вносят пространственную зависимость, делая сеть чувствительной к точному положению входных паттернов.

В последние годы было предложено несколько подходов к решению проблемы пространственной вариативности CNN, включая методы анти-алиасинга (например, BlurPool), полифазную выборку с инвариантностью к сдвигам (TIPS) и различные модификации архитектур. Однако систематическое исследование влияния этих методов на стабильность работы различных типов CNN в контексте разных задач компьютерного зрения остается актуальной проблемой.

Данная работа направлена на всестороннее исследование артефактов пространственной инвариантности в современных CNN-архитектурах, анализ их влияния на производительность моделей и оценку эффективности различных методов повышения устойчивости к пространственным сдвигам. Особое внимание уделяется сравнению поведения классификационных моделей и моделей детекции объектов, таких как YOLO, при субпиксельных сдвигах входных данных, что позволяет выявить специфические проблемы и предложить целевые решения для различных типов архитектур.

\section*{Цель и задачи исследования}
\label{intro:goal}

\textbf{Целью} данной работы является комплексное исследование проблемы отсутствия полной инвариантности к пространственным сдвигам в современных архитектурах сверточных нейронных сетей, разработка и оценка методов повышения их устойчивости к смещениям входных данных.

Для достижения поставленной цели необходимо решить следующие \textbf{задачи}:

\begin{enumerate}
    \item Провести анализ существующих исследований и методов в области пространственной инвариантности CNN, включая:
    \begin{itemize}
        \item Теоретические основы инвариантности к сдвигам в сверточных архитектурах
        \item Методы анти-алиасинга в нейронных сетях
        \item Подходы к обеспечению инвариантности в моделях детекции объектов
        \item Техники полифазной выборки с инвариантностью к сдвигам
    \end{itemize}
    
    \item Формализовать проблему пространственной инвариантности и разработать математическую модель для описания влияния операций субдискретизации на стабильность представлений в CNN.
    
    \item Разработать методологию тестирования и метрики для количественной оценки степени инвариантности моделей к пространственным сдвигам, включая:
    \begin{itemize}
        \item Методику генерации последовательностей изображений с контролируемыми субпиксельными сдвигами
        \item Метрики стабильности векторов признаков (косинусное сходство)
        \item Метрики стабильности предсказаний (дрейф уверенности, стабильность IoU)
        \item Визуализации для качественного анализа эффектов
    \end{itemize}
    
    \item Провести экспериментальное исследование влияния субпиксельных сдвигов на стабильность работы различных CNN-архитектур:
    \begin{itemize}
        \item Классификационных моделей (VGG16, ResNet50)
        \item Моделей детекции объектов (YOLOv5)
        \item Их модифицированных версий с различными методами повышения инвариантности
    \end{itemize}
    
    \item Реализовать и сравнить различные методы повышения инвариантности к сдвигам:
    \begin{itemize}
        \item Классический анти-алиасинг (BlurPool)
        \item Translation Invariant Polyphase Sampling (TIPS)
        \item Гибридные подходы
    \end{itemize}
    
    \item Провести аблационное исследование для выявления влияния различных факторов на пространственную инвариантность:
    \begin{itemize}
        \item Размера рецептивного поля
        \item Разных типов операций пулинга
        \item Параметров анти-алиасинга
    \end{itemize}
    
    \item Сформулировать практические рекомендации по выбору архитектур и методов обеспечения инвариантности для различных задач компьютерного зрения.
\end{enumerate}

\section*{Научная новизна и практическая значимость}
\label{intro:novelty}

\textbf{Научная новизна} данной работы заключается в следующем:

\begin{enumerate}
    \item Проведено комплексное сравнительное исследование проблемы пространственной инвариантности в различных типах CNN-архитектур (классификаторы и детекторы) с использованием единой методологии и системы метрик.
    
    \item Разработана и апробирована методика генерации контролируемых последовательностей изображений с субпиксельными сдвигами, позволяющая точно измерять степень инвариантности моделей к пространственным преобразованиям.
    
    \item Предложены новые метрики и визуализации для количественной и качественной оценки стабильности работы CNN при пространственных сдвигах входных данных.
    
    \item Впервые проведено систематическое сравнение эффективности различных методов обеспечения инвариантности (BlurPool, TIPS) в контексте моделей детекции объектов (YOLOv5).
    
    \item Проведено аблационное исследование, позволяющее выявить ключевые факторы, влияющие на степень пространственной инвариантности в современных CNN.
\end{enumerate}

\textbf{Практическая значимость} работы определяется следующими аспектами:

\begin{enumerate}
    \item Результаты исследования позволяют более осознанно подходить к выбору архитектур CNN для задач, требующих высокой стабильности предсказаний при малых изменениях входных данных.
    
    \item Предложенные модификации архитектур с использованием методов анти-алиасинга и полифазной выборки могут быть непосредственно применены для улучшения стабильности существующих систем компьютерного зрения.
    
    \item Разработанная методология тестирования и система метрик могут использоваться как инструментарий для оценки пространственной инвариантности при разработке новых архитектур нейронных сетей.
    
    \item Сформулированные рекомендации по выбору методов обеспечения инвариантности имеют практическую ценность для разработчиков систем компьютерного зрения в таких областях, как:
    \begin{itemize}
        \item Автономные транспортные средства и роботы, где стабильность детекции объектов критически важна для безопасности
        \item Медицинская визуализация, где точность локализации патологий напрямую влияет на качество диагностики
        \item Системы видеоаналитики, требующие надежного отслеживания объектов при их перемещении
        \item Промышленные системы контроля качества, где незначительные изменения положения контролируемых объектов не должны влиять на результаты анализа
    \end{itemize}
\end{enumerate}

\section*{Структура работы}
\label{intro:structure}

Диссертация состоит из введения, трех глав, заключения, списка литературы и приложений. Общий объем работы составляет XX страниц, включая XX рисунков и XX таблиц. Список литературы содержит XX наименований.

\textbf{В главе 1} представлен обзор литературы по проблеме пространственной инвариантности в сверточных нейронных сетях. Рассмотрены теоретические основы инвариантности к сдвигам, проанализированы причины нарушения этого свойства в современных CNN-архитектурах, описаны существующие методы повышения устойчивости к пространственным преобразованиям. Особое внимание уделено специфике проблемы инвариантности в моделях детекции объектов.

\textbf{В главе 2} изложены теоретические основы исследования. Формализована проблема пространственной инвариантности, представлен математический аппарат для описания влияния операций субдискретизации на стабильность представлений в CNN. Подробно рассмотрены архитектуры исследуемых моделей (VGG16, ResNet50, YOLOv5) и методы повышения их инвариантности к сдвигам (BlurPool, TIPS). Приведен детальный анализ рецептивных полей в различных архитектурах и их связи с проблемой пространственной инвариантности.

\textbf{В главе 3} описана экспериментальная часть исследования. Представлена методология тестирования, включая генерацию контрольных последовательностей изображений с субпиксельными сдвигами, определены используемые метрики, детально описан процесс проведения экспериментов. Приведены результаты сравнительного анализа различных архитектур и методов повышения инвариантности, представлены визуализации, демонстрирующие эффекты пространственных сдвигов на работу моделей. Проведен анализ производительности модифицированных архитектур и оценка компромисса между вычислительной сложностью и стабильностью предсказаний.

\textbf{В заключении} обобщены основные результаты работы, сформулированы выводы и рекомендации по выбору архитектур и методов обеспечения инвариантности для различных задач компьютерного зрения, а также обозначены перспективные направления дальнейших исследований в данной области.