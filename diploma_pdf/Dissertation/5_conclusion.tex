\chapter*{Заключение}
\addcontentsline{toc}{chapter}{Заключение}

В настоящей работе было проведено комплексное исследование проблемы пространственной инвариантности в современных архитектурах сверточных нейронных сетей и разработаны методы повышения их устойчивости к смещениям входных данных. Основные результаты работы заключаются в следующем:

\begin{enumerate}
    \item Проведен анализ существующих исследований и методов в области пространственной инвариантности CNN, выявлены основные причины нарушения инвариантности к сдвигам, связанные с операциями субдискретизации (даунсэмплинга) в современных архитектурах. Установлено, что наиболее критичными факторами являются операции max-pooling и свёртка с шагом больше единицы, которые нарушают теорему о дискретизации Найквиста-Шеннона.
    
    \item Формализована проблема пространственной инвариантности и разработана математическая модель для описания влияния операций субдискретизации на стабильность представлений в CNN. Показано, что без соответствующей фильтрации низких частот перед даунсэмплингом неизбежно возникает эффект алиасинга, приводящий к нестабильности предсказаний при малых сдвигах входных данных.
    
    \item Разработана методология тестирования и система метрик для количественной оценки степени инвариантности моделей к пространственным сдвигам, включающая:
    \begin{itemize}
        \item Алгоритм генерации последовательностей изображений с контролируемыми субпиксельными сдвигами, позволяющий точно измерять влияние смещений на различные аспекты работы модели
        \item Метрики стабильности векторов признаков (косинусное сходство)
        \item Метрики стабильности предсказаний (дрейф уверенности для классификаторов, стабильность IoU и позиции центра для детекторов)
        \item Инструменты визуализации для качественного анализа эффектов смещения
    \end{itemize}
    
    \item Проведено экспериментальное исследование влияния субпиксельных сдвигов на стабильность работы различных CNN-архитектур, включая классификационные модели (VGG16, ResNet50) и модели детекции объектов (YOLOv5). Экспериментально показано, что:
    \begin{itemize}
        \item Стандартные CNN-архитектуры демонстрируют значительную нестабильность даже при минимальных (субпиксельных) сдвигах входных данных
        \item Степень нестабильности зависит от глубины сети, количества операций субдискретизации и размера рецептивного поля
        \item Модели детекции объектов особенно чувствительны к смещениям, что проявляется в значительном дрейфе предсказанных ограничивающих рамок и падении показателя IoU
    \end{itemize}
    
    \item Реализованы и сравнены различные методы повышения инвариантности к сдвигам:
    \begin{itemize}
        \item Классический анти-алиасинг с использованием BlurPool, который заменяет стандартные операции пулинга на последовательность фильтра низких частот и операции субдискретизации
        \item Translation Invariant Polyphase Sampling (TIPS), который использует полифазное разложение для сохранения информации при даунсэмплинге
        \item Гибридные подходы, комбинирующие различные методы в зависимости от структуры сети
    \end{itemize}
    
    \item Проведено аблационное исследование для выявления влияния различных факторов на пространственную инвариантность, которое показало, что:
    \begin{itemize}
        \item Увеличение размера рецептивного поля улучшает инвариантность к сдвигам, но не решает проблему полностью
        \item Замена max-pooling на average-pooling снижает нестабильность, но все еще требует анти-алиасинга для достижения высокой инвариантности
        \item Оптимальный размер ядра для анти-алиасинга составляет 3×3 или 5×5, обеспечивая баланс между инвариантностью и вычислительной сложностью
    \end{itemize}
\end{enumerate}

На основе проведенного исследования могут быть сформулированы следующие практические рекомендации:

\begin{enumerate}
    \item Для задач, требующих высокой стабильности предсказаний при малых изменениях входных данных (например, системы автономного вождения, медицинская диагностика), рекомендуется использовать модифицированные архитектуры с методами анти-алиасинга, предпочтительно с применением TIPS в критических слоях.
    
    \item При ограниченных вычислительных ресурсах оптимальным выбором является применение BlurPool с ядром 3×3 в последних слоях сверточной части сети, что обеспечивает значительное повышение инвариантности при минимальном увеличении вычислительной сложности.
    
    \item Для моделей детекции объектов, особенно YOLOv5, рекомендуется модификация backbone-сети с применением техник анти-алиасинга, что существенно улучшает стабильность позиционирования ограничивающих рамок при сдвигах объектов.
    
    \item При обучении новых моделей рекомендуется включать в набор данных аугментации с субпиксельными сдвигами, что повышает робастность модели к подобным преобразованиям даже без структурных изменений архитектуры.
\end{enumerate}

Полученные результаты имеют как теоретическую, так и практическую значимость для разработки более надежных систем компьютерного зрения. Предложенные методики оценки инвариантности и модификации архитектур могут быть непосредственно применены при создании и оптимизации CNN для различных прикладных задач.

Направления дальнейших исследований могут включать:
\begin{itemize}
    \item Разработку методов обеспечения инвариантности к более широкому классу геометрических преобразований (поворот, масштабирование, аффинные преобразования)
    \item Исследование проблемы инвариантности в современных трансформерных архитектурах для компьютерного зрения
    \item Разработку специализированных аппаратных решений для эффективной реализации методов анти-алиасинга и полифазной выборки
    \item Интеграцию методов обеспечения инвариантности в фреймворки автоматического поиска архитектур нейронных сетей
\end{itemize}

Таким образом, проведенное исследование не только вносит вклад в понимание фундаментальной проблемы пространственной инвариантности в CNN, но и предлагает практические решения, повышающие надежность и стабильность систем компьютерного зрения в различных прикладных областях.