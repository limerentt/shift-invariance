\chapter{Заключение} \label{conclusion}

В данной работе проведено комплексное исследование проблемы пространственной инвариантности в свёрточных нейронных сетях (CNN) и методов её повышения. Исследование включало как теоретический анализ причин недостаточной инвариантности к сдвигам, так и экспериментальное сравнение различных архитектур и методов анти-алиасинга.

\section{Основные результаты работы}
\label{conclusion:results}

В рамках проведенного исследования были получены следующие основные результаты:

\begin{enumerate}
    \item \textbf{Теоретически обоснованы причины недостаточной инвариантности CNN к сдвигам.} Показано, что ключевой причиной этой проблемы является алиасинг, возникающий при операциях даунсэмплинга (максимальный пулинг, свертки с шагом больше 1), который нарушает свойство эквивариантности сетевых представлений к пространственным трансформациям входа.
    
    \item \textbf{Разработана методология количественной оценки инвариантности к сдвигам.} Предложен комплекс метрик для оценки стабильности как внутренних представлений (косинусное сходство векторов признаков), так и выходных предсказаний (дрейф уверенности, стабильность класса) для задач классификации и детекции объектов.
    
    \item \textbf{Экспериментально подтверждена и количественно оценена проблема недостаточной инвариантности к сдвигам.} Для базовых архитектур (VGG16, ResNet50, YOLOv5) продемонстрировано значительное варьирование внутренних представлений и предсказаний при субпиксельных сдвигах входных изображений, что подтверждает практическую значимость проблемы.
    
    \item \textbf{Проведено сравнительное исследование методов анти-алиасинга.} Показано, что методы BlurPool и TIPS существенно повышают инвариантность CNN к сдвигам, с наилучшими результатами для метода TIPS, обеспечивающего почти идеальную инвариантность с минимальным дрейфом представлений и предсказаний.
    
    \item \textbf{Проведены аблационные эксперименты для выявления факторов, влияющих на инвариантность.} Установлено, что размер рецептивного поля, тип архитектуры и характеристики фильтров влияют на степень инвариантности, но не устраняют проблему алиасинга без специальных методов анти-алиасинга.
    
    \item \textbf{Выполнено профилирование производительности методов анти-алиасинга.} Количественно оценено влияние различных методов на вычислительную эффективность, потребление памяти и энергии, что позволяет обоснованно выбирать оптимальный метод для конкретного применения.
    
    \item \textbf{Сформулированы практические рекомендации по применению методов анти-алиасинга} для различных сценариев использования и отраслевых приложений, с учетом компромисса между инвариантностью и вычислительной эффективностью.
\end{enumerate}

Полученные результаты имеют как теоретическую значимость для понимания свойств глубоких нейронных сетей, так и практическую ценность для разработки более стабильных и надежных систем компьютерного зрения.

\section{Научная новизна}
\label{conclusion:novelty}

Научная новизна работы состоит в следующем:

\begin{enumerate}
    \item Впервые проведено комплексное сравнительное исследование методов анти-алиасинга (BlurPool и TIPS) на единой методологической основе с использованием разнообразных количественных метрик инвариантности.
    
    \item Разработана и апробирована методология оценки инвариантности к сдвигам для задач детекции объектов, включающая анализ стабильности ограничивающих рамок и визуализацию с помощью динамических последовательностей.
    
    \item Проведено исследование взаимосвязи между размером рецептивного поля и инвариантностью к сдвигам, демонстрирующее, что увеличение рецептивного поля не решает проблему алиасинга, но может частично компенсировать его эффекты.
    
    \item Впервые выполнено систематическое профилирование методов анти-алиасинга на различных аппаратных платформах, включая встраиваемые системы, что имеет практическую ценность для реальных приложений.
    
    \item Предложены рекомендации по оптимальному выбору и интеграции методов анти-алиасинга в существующие архитектуры и системы, учитывающие специфику различных отраслевых применений.
\end{enumerate}

\section{Практическая значимость}
\label{conclusion:significance}

Практическая значимость полученных результатов определяется следующими аспектами:

\begin{enumerate}
    \item \textbf{Повышение стабильности систем компьютерного зрения.} Применение методов анти-алиасинга позволяет существенно повысить стабильность предсказаний при малых пространственных сдвигах объектов, что критично для многих практических применений, включая трекинг объектов, медицинскую визуализацию и системы автономного управления.
    
    \item \textbf{Оптимизация выбора архитектуры CNN.} Проведенный анализ позволяет обоснованно выбирать архитектуру сети и метод анти-алиасинга в зависимости от требований к инвариантности и вычислительной эффективности для конкретной задачи.
    
    \item \textbf{Применимость для встраиваемых систем.} Показано, что даже на устройствах с ограниченными вычислительными ресурсами (Jetson Xavier NX) методы анти-алиасинга обеспечивают приемлемую производительность при значительном улучшении инвариантности, что расширяет область их практического применения.
    
    \item \textbf{Практические рекомендации.} Сформулированные рекомендации позволяют практикам выбирать оптимальный подход к повышению инвариантности в зависимости от специфики конкретной задачи и доступных вычислительных ресурсов.
    
    \item \textbf{Методика оценки инвариантности.} Разработанная методика может использоваться для оценки качества различных моделей компьютерного зрения с точки зрения их инвариантности к пространственным трансформациям.
\end{enumerate}

\section{Перспективы дальнейших исследований}
\label{conclusion:perspectives}

На основе результатов данной работы можно наметить следующие перспективные направления дальнейших исследований:

\begin{enumerate}
    \item \textbf{Расширение на другие типы инвариантности.} Исследование инвариантности к другим типам трансформаций (масштабированию, повороту, аффинным преобразованиям) и разработка методов, обеспечивающих комплексную пространственную инвариантность.
    
    \item \textbf{Применение к современным архитектурам.} Адаптация и оценка эффективности методов анти-алиасинга для современных архитектур, таких как Vision Transformer, MLP-Mixer, ConvNext и другие.
    
    \item \textbf{Интеграция с методами оптимизации моделей.} Исследование взаимодействия методов анти-алиасинга с техниками оптимизации моделей (квантизация, прунинг, дистилляция) и разработка подходов, сохраняющих инвариантность при оптимизации.
    
    \item \textbf{Применение в других областях.} Исследование эффективности методов анти-алиасинга в других областях глубокого обучения, таких как обработка медицинских изображений, спутниковых снимков, видеоаналитика и других.
    
    \item \textbf{Аппаратные реализации.} Разработка эффективных аппаратных реализаций методов анти-алиасинга для специализированных нейросетевых ускорителей, оптимизированных для энергоэффективности и низкой задержки.
\end{enumerate}

\section{Заключительные замечания}
\label{conclusion:final}

Проблема пространственной инвариантности в CNN имеет как фундаментальное теоретическое значение для понимания свойств глубоких нейронных сетей, так и существенные практические импликации для разработки надежных систем компьютерного зрения. Результаты данной работы вносят вклад в оба эти аспекта, предоставляя как теоретическое обоснование причин недостаточной инвариантности, так и практические рекомендации по её повышению.

Методы анти-алиасинга, особенно TIPS, демонстрируют значительный потенциал для улучшения стабильности современных систем компьютерного зрения при относительно небольших вычислительных затратах. Их широкое внедрение может способствовать созданию более надежных и предсказуемых систем искусственного интеллекта для различных практических применений, от автономных транспортных средств и роботов до систем медицинской диагностики и промышленного контроля качества.

В целом, результаты исследования подтверждают, что архитектурные решения, основанные на принципах обработки сигналов и учитывающие фундаментальные ограничения дискретизации, могут существенно улучшить свойства глубоких нейронных сетей без необходимости в дополнительном обучении или значительном увеличении вычислительной сложности. Это открывает перспективы для более тесной интеграции классической теории обработки сигналов и современных методов глубокого обучения в будущих исследованиях. 