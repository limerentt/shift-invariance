%%% Основные сведения %%%
\newcommand{\thesisAuthor}             % Диссертация, ФИО автора
{%
    \texorpdfstring{% \texorpdfstring takes two arguments and uses the first for (La)TeX and the second for pdf
        Полев Алексей Михайлович % так будет отображаться на титульном листе или в тексте, где будет использоваться переменная
    }{%
        Полев Алексей Михайлович % эта запись для свойств pdf-файла. В таком виде, если pdf будет обработан программами для сбора библиографических сведений, будет правильно представлена фамилия.
    }%
}

\newcommand{\thesisTitle}              % Диссертация, название
{\texorpdfstring{\MakeUppercase{Симуляция сдвига домена и анализ изменения промежуточных низкоразмерных представлений в ResNet-подобных архитектурах для задачи компьютерного зрения}}{Симуляция сдвига домена и анализ изменения промежуточных низкоразмерных представлений в ResNet-подобных архитектурах для задачи компьютерного зрения}}
\newcommand{\thesisSpecialtyNumber}    % Диссертация, специальность, номер
{\texorpdfstring{01.03.02}{01.03.02}}
\newcommand{\thesisSpecialtyTitle}     % Диссертация, специальность, название
{\texorpdfstring{Прикладная математика и информатика}{Прикладная математика и информатика}}
\newcommand{\thesisDegree}             % Диссертация, научная степень
{} % Убрана информация о магистерской диссертации согласно указанию в чек-листе
\newcommand{\thesisCity}               % Диссертация, город защиты
{Москва}
\newcommand{\thesisYear}               % Диссертация, год защиты
{2025}
\newcommand{\thesisOrganization}       % Диссертация, организация
{Федеральное государственное автономное образовательное учреждение
 высшего образования
 «Московский физико-технический институт
(национальный исследовательский университет)»}

\newcommand{\supervisorFio}            % Научный руководитель, ФИО
{Самосюк Алексей Владимирович}

\newcommand{\thesisUdk}                % УДК диссертации
{004.93'12}
\newcommand{\keywords}                 % Ключевые слова для метаданных PDF диссертации и автореферата
{CNN, сверточные нейронные сети, пространственная инвариантность, анти-алиасинг, 
TIPS, компьютерное зрение, BlurPool, YOLOv5}