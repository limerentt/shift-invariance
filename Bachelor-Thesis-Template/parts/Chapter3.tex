\section{Исследование и построение решения задачи}
\label{sec:Chapter3} \index{Chapter3}

В данной главе описывается исследовательская часть работы, основанная на анализе литературы, представленном в предыдущей главе. Рассматриваются методы улучшения инвариантности к сдвигам в современных нейронных сетях и предлагаемые модификации архитектур.

\subsection{Математическая формализация и модификации архитектур}
\label{sec:math}

Для формализации проблемы инвариантности к сдвигам рассмотрим нейронную сеть как функцию $f: \mathbb{R}^{H \times W \times C} \rightarrow \mathbb{R}^K$. Для операции сдвига $\mathcal{T}_{\Delta h, \Delta w}$, инвариантность к сдвигам означает, что $f(\mathcal{T}_{\Delta h, \Delta w}(x)) \approx f(x)$. Для оценки инвариантности используется метрика стабильности предсказаний при субпиксельных сдвигах, измеряющая среднее сходство выходов модели для оригинального и сдвинутых изображений.

В работе исследуются две основные методики борьбы с алиасингом: BlurPool и TIPS, которые были адаптированы для различных архитектур нейронных сетей.

\subsubsection{BlurPool в классификационных моделях}
\label{sec:architectures:blurpool_classification}

Для классификационных моделей VGG16 и ResNet50 реализованы модификации с применением BlurPool, заключающиеся в замене операций даунсэмплинга на их аналоги с предварительной низкочастотной фильтрацией:

$$\text{BlurPool}_{m,s} = \text{Subsample}_{s} \circ \text{Blur}_{m}$$

где $\text{Subsample}_{s}$ — операция субдискретизации с шагом $s$, а $\text{Blur}_{m}$ — операция свёртки с фиксированным низкочастотным фильтром размера $m \times m$.

Используются биномиальные фильтры:
\begin{itemize}
    \item \textbf{Triangle-3}: [1, 2, 1] × [1, 2, 1]$^{T}$ / 16
    \item \textbf{Binomial-5}: [1, 4, 6, 4, 1] × [1, 4, 6, 4, 1]$^{T}$ / 256
\end{itemize}

Модификации применяются следующим образом:
\begin{itemize}
    \item \textbf{MaxPool} $\to$ $\text{Subsample}_{s} \circ \text{Blur}_{m} \circ \text{Max}_{k,1}$
    \item \textbf{Свертка с шагом} $\to$ $\text{BlurPool}_{m,s} \circ \text{ReLU} \circ \text{Conv}_{k,1}$
    \item \textbf{AvgPool} $\to$ $\text{BlurPool}_{m,s}$
\end{itemize}

Важное преимущество метода BlurPool — возможность применения к предобученным моделям с минимальным увеличением вычислительных затрат (<1%).

\subsubsection{TIPS для повышения инвариантности}
\label{sec:architectures:tips}

TIPS (Translation Invariant Polyphase Sampling) — более продвинутый подход, основанный на разделении сигнала на несколько фаз в зависимости от его положения относительно сетки субдискретизации:

\begin{equation}
\text{TIPS}(x) = \frac{1}{s^2}\sum_{i=0}^{s-1}\sum_{j=0}^{s-1} \text{Subsample}_{s}(\text{Shift}_{(i,j)}(x))
\end{equation}

где $s$ - коэффициент субдискретизации, а $\text{Shift}_{(i,j)}$ - операция сдвига.

В реализации TIPS для слоя с шагом $s$ создаются $s^2$ отдельных ветвей, каждая обрабатывает сдвинутую версию входного тензора, результаты объединяются для формирования инвариантного представления.

\subsection{Архитектура YOLOv5 и её модификации}
\label{sec:yolov5}

Особое внимание уделяется детектору YOLOv5, состоящему из трех компонентов:
\begin{itemize}
    \item Backbone (CSPDarknet) — извлекает признаки
    \item Neck (PANet) — объединяет признаки разных уровней
    \item Head — преобразует признаки в предсказания
\end{itemize}

Модификации затрагивают операции даунсэмплинга в backbone и neck:
\begin{itemize}
    \item YOLOv5-BlurPool: замена сверток с шагом 2 на свертку с шагом 1 + BlurPool
    \item YOLOv5-TIPS: замена сверток с шагом 2 на TIPS-модули
\end{itemize}

От модификаций ожидаются: повышение стабильности предсказаний, уменьшение дрейфа центра ограничивающей рамки, более высокая стабильность IoU между предсказанными и истинными рамками.

\subsection{Методология оценки инвариантности}
\label{sec:evaluation}

\subsubsection{Метрики для классификационных моделей}
\label{sec:evaluation:classification}

Для классификационных моделей используются:

\begin{itemize}
    \item \textbf{Top-1 Accuracy (Acc)}: доля правильно классифицированных изображений.
    
    \item \textbf{Consistency (Cons)}: вероятность одинакового предсказания для исходного и сдвинутого изображения:
    \begin{equation}
    \text{Cons} = \mathbb{E}_{x, \delta} \left[ \mathbb{I} \left( \underset{c}{\text{argmax}} \, f(x)_c = \underset{c}{\text{argmax}} \, f(\mathcal{T}_\delta(x))_c \right) \right]
    \end{equation}
    
    \item \textbf{Stability (Stab)}: среднее косинусное сходство между выходными представлениями:
    \begin{equation}
    \text{Stab} = \mathbb{E}_{x, \delta} \left[ \frac{f(x) \cdot f(\mathcal{T}_\delta(x))}{\|f(x)\| \cdot \|f(\mathcal{T}_\delta(x))\|} \right]
    \end{equation}
\end{itemize}

\subsubsection{Метрики для детекторов объектов}
\label{sec:evaluation:detection}

Для детекторов объектов используются:

\begin{itemize}
    \item \textbf{mAP}: средняя точность детекции при различных порогах IoU.
    
    \item \textbf{IoU Stability (IS)}: стабильность пересечения над объединением рамок при сдвигах:
    \begin{equation}
    \text{IS} = \mathbb{E}_{x, \delta, b} \left[ \text{IoU} \left( b, \mathcal{T}_{-\delta}(b_{\delta}) \right) \right]
    \end{equation}
    где $b$ — ограничивающая рамка для исходного изображения, $b_{\delta}$ — соответствующая рамка для сдвинутого изображения, а $\mathcal{T}_{-\delta}$ — обратный сдвиг для компенсации смещения.
    
    \item \textbf{Center Drift (CD)}: среднее смещение центров рамок при сдвигах:
    \begin{equation}
    \text{CD} = \mathbb{E}_{x, \delta, b} \left[ \| \text{center}(b) - \text{center}(\mathcal{T}_{-\delta}(b_{\delta})) \|_2 \right]
    \end{equation}
\end{itemize}

\subsubsection{Протокол тестирования}
\label{sec:evaluation:protocol}

Протокол тестирования включает:

\begin{enumerate}
    \item \textbf{Генерация сдвигов}: Для каждого изображения создается набор сдвинутых версий с точностью до 1/8 пикселя в диапазоне $[-8, 8]$ пикселей.
    
    \item \textbf{Предобработка}: Стандартизация размера до 224×224 пикселей для классификации и 640×640 для детекции.
    
    \item \textbf{Проверка инвариантности}: Сравнение выходов модели для оригинального и сдвинутых изображений.
    
    \item \textbf{Агрегация результатов}: Усреднение метрик по всем изображениям и сдвигам.
\end{enumerate}

Этот подход позволяет провести исчерпывающую оценку инвариантности моделей и сравнить эффективность различных методов анти-алиасинга в задачах классификации и детекции объектов.

\newpage
