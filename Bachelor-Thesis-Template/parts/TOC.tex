\newpage
\section*{СОДЕРЖАНИЕ}
\thispagestyle{empty}

\tableofcontents
\newpage

% Этот файл создаёт структурированное содержание дипломной работы.
% LaTeX автоматически соберёт оглавление на основе заголовков разделов в документе.

% Структура работы:
% - АННОТАЦИЯ
% - СПИСОК СОКРАЩЕНИЙ И ОБОЗНАЧЕНИЙ
% - ВВЕДЕНИЕ
%   * Описание проблемы shift-invariance
%   * Актуальность исследования
%   * Обзор существующих подходов
%   * Цель исследования
% - 1. ОБЗОР ЛИТЕРАТУРЫ
%   * 1.1. Свёрточные нейронные сети
%   * 1.2. Проблема shift-invariance
%   * 1.3. Существующие методы решения проблемы
%   * 1.4. Методы оценки shift-invariance
% - 2. ТЕОРЕТИЧЕСКАЯ ЧАСТЬ И МЕТОДОЛОГИЯ
%   * 2.1. Формализация проблемы shift-invariance
%   * 2.2. Предлагаемые методы решения
%   * 2.3. Математическое обоснование
% - 3. ЭКСПЕРИМЕНТАЛЬНАЯ ЧАСТЬ
%   * 3.1. Используемые датасеты
%   * 3.2. Архитектуры моделей
%   * 3.3. Метрики оценки
%   * 3.4. Условия проведения экспериментов
%   * 3.5. Реализация методов
% - 4. РЕЗУЛЬТАТЫ
%   * 4.1. Сравнение с базовыми моделями
%   * 4.2. Влияние предложенных методов на shift-invariance
%   * 4.3. Анализ результатов на задачах классификации
%   * 4.4. Анализ результатов на задачах детекции
%   * 4.5. Обсуждение результатов
% - ЗАКЛЮЧЕНИЕ
% - СПИСОК ЛИТЕРАТУРЫ
% - ПРИЛОЖЕНИЯ (при необходимости)

% Содержание будет автоматически сгенерировано из заголовков разделов и подразделов
% Ниже приведена структура работы для справки:

% АННОТАЦИЯ

% СПИСОК СОКРАЩЕНИЙ И ОБОЗНАЧЕНИЙ

% ВВЕДЕНИЕ
%   - Описание проблемы shift-invariance
%   - Актуальность исследования
%   - Обзор существующих подходов
%   - Цель исследования

% 1. ОБЗОР ЛИТЕРАТУРЫ
%   1.1. Свёрточные нейронные сети
%   1.2. Проблема shift-invariance
%   1.3. Существующие методы решения проблемы
%   1.4. Методы оценки shift-invariance

% 2. ТЕОРЕТИЧЕСКАЯ ЧАСТЬ И МЕТОДОЛОГИЯ
%   2.1. Формализация проблемы shift-invariance
%   2.2. Предлагаемые методы решения
%   2.3. Математическое обоснование

% 3. ЭКСПЕРИМЕНТАЛЬНАЯ ЧАСТЬ
%   3.1. Используемые датасеты
%   3.2. Архитектуры моделей
%   3.3. Метрики оценки
%   3.4. Условия проведения экспериментов
%   3.5. Реализация методов

% 4. РЕЗУЛЬТАТЫ
%   4.1. Сравнение с базовыми моделями
%   4.2. Влияние предложенных методов на shift-invariance
%   4.3. Анализ результатов на задачах классификации
%   4.4. Анализ результатов на задачах детекции
%   4.5. Обсуждение результатов

% ЗАКЛЮЧЕНИЕ

% СПИСОК ЛИТЕРАТУРЫ

% ПРИЛОЖЕНИЯ (при необходимости) 