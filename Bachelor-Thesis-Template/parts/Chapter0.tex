\section*{Введение}
\label{sec:Chapter0} \index{Chapter0}
\addcontentsline{toc}{section}{Введение}

\section*{Актуальность проблемы}
\label{intro:relevance}

Сверточные нейронные сети (CNN) сегодня являются ключевым инструментом в решении широкого спектра задач компьютерного зрения, включая классификацию изображений \cite{krizhevsky2012imagenet, simonyan2014very}, сегментацию \cite{long2015fully}, детекцию объектов \cite{redmon2016you, ren2015faster} и другие. Их популярность и эффективность обусловлены способностью к автоматическому извлечению иерархии признаков из необработанных данных и высокой точностью работы в различных условиях. Теоретические основы CNN предполагают, что они должны обладать свойством инвариантности к пространственным преобразованиям, в частности, к сдвигам входных данных \cite{lecun1998gradient}. Это означает, что одинаковые объекты, расположенные в разных частях изображения, должны распознаваться с одинаковой точностью и уверенностью.

Однако практика показывает, что современные архитектуры CNN не обладают полной инвариантностью к сдвигам \cite{zhang2019making, azulay2019deep}. Небольшие, даже субпиксельные смещения объектов на входном изображении могут приводить к значительным изменениям в выходных результатах сети. Эта проблема, часто упускаемая из виду при традиционной оценке моделей на тестовых выборках, может иметь серьезные последствия в реальных приложениях компьютерного зрения, особенно в критически важных областях, таких как автономные транспортные средства, системы видеонаблюдения, медицинская диагностика и робототехника.

Отсутствие стабильности предсказаний при малых смещениях объектов может привести к:
\begin{itemize}
    \item Ложным срабатываниям или пропускам в системах обнаружения объектов
    \item Нестабильной работе алгоритмов слежения за объектами
    \item Некорректной сегментации медицинских изображений
    \item Ошибкам в системах управления роботами и беспилотными автомобилями
    \item Снижению надежности систем биометрической идентификации
\end{itemize}

Причины нарушения инвариантности к сдвигам в CNN связаны с операциями субдискретизации (даунсэмплинга), такими как max-pooling и свёртка с шагом (stride) больше единицы \cite{zhang2019making}. Эти операции позволяют уменьшать пространственное разрешение карт признаков, что необходимо для снижения вычислительной сложности и обобщающей способности сети, но одновременно вносят пространственную зависимость, делая сеть чувствительной к точному положению входных паттернов. С точки зрения теории сигналов, эта проблема связана с нарушением теоремы Найквиста-Шеннона при дискретизации, что приводит к эффекту алиасинга \cite{shannon1949communication}.

В последние годы было предложено несколько подходов к решению проблемы пространственной вариативности CNN, включая методы анти-алиасинга (например, BlurPool \cite{zhang2019making}), полифазную выборку с инвариантностью к сдвигам (TIPS \cite{chaman2021truly}) и различные модификации архитектур \cite{kayhan2020translation}. Однако систематическое исследование влияния этих методов на стабильность работы различных типов CNN в контексте разных задач компьютерного зрения остается актуальной проблемой.

Данная работа направлена на всестороннее исследование артефактов пространственной инвариантности в современных CNN-архитектурах, анализ их влияния на производительность моделей и оценку эффективности различных методов повышения устойчивости к пространственным сдвигам. Особое внимание уделяется сравнению поведения классификационных моделей и моделей детекции объектов, таких как YOLO, при субпиксельных сдвигах входных данных, что позволяет выявить специфические проблемы и предложить целевые решения для различных типов архитектур. В рамках данного исследования впервые реализован метод полифазной выборки с инвариантностью к сдвигам (TIPS) для моделей семейства YOLO, который продемонстрировал значительно более высокую стабильность результатов детекции при пространственных сдвигах по сравнению с классической версией этой же модели.

\newpage
