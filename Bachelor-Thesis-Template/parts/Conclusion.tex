\section{Заключение}
\addcontentsline{toc}{section}{Заключение}
\label{sec:Conclusion} \index{Conclusion}

В данной работе был проведен комплексный анализ проблемы инвариантности к сдвигам в сверточных нейронных сетях и исследованы методы повышения их устойчивости к субпиксельным сдвигам входных изображений. Экспериментальные исследования подтвердили, что применение методов антиалиасинга (BlurPool и TIPS) значительно улучшает инвариантность к сдвигам как в задачах классификации, так и детекции объектов, повышая при этом точность и стабильность предсказаний при минимальных вычислительных издержках.

На основе полученных результатов можно сформулировать следующие практические рекомендации по выбору и применению методов повышения инвариантности к сдвигам в различных прикладных задачах:

\begin{enumerate}
    \item \textbf{Для высокоточных систем критического применения} (медицинская диагностика, системы безопасности, автономное вождение) рекомендуется использовать метод TIPS, который обеспечивает максимальную стабильность предсказаний (повышение консистентности до 97-98\%) и наименьший дрейф локализации объектов. Несмотря на увеличение вычислительных затрат на 4-5\%, критически важная стабильность результатов полностью оправдывает это незначительное снижение производительности.
    
    \item \textbf{Для систем с ограниченными вычислительными ресурсами} (мобильные приложения, встраиваемые системы, обработка видеопотока в реальном времени) оптимальным выбором является метод BlurPool, который обеспечивает существенное повышение инвариантности (консистентность 91-94\%) при минимальном увеличении вычислительной нагрузки (менее 1\% дополнительных FLOPs). Это решение особенно эффективно для устройств с батарейным питанием и ограниченной вычислительной мощностью.
    
    \item \textbf{Для задач детекции малоразмерных объектов} особенно рекомендуется применение методов антиалиасинга, поскольку эксперименты показали, что наибольший выигрыш в точности (до 2.6 процентных пунктов) достигается именно для малых объектов, которые наиболее чувствительны к сдвигам из-за ограниченного количества представляющих их пикселей.
\end{enumerate}

Внедрение предложенных методов не требует значительных изменений в архитектуре существующих моделей и может быть реализовано как простая замена операций субдискретизации, что делает их практическое применение доступным для широкого круга задач компьютерного зрения. Повышение инвариантности к сдвигам является важным шагом на пути к созданию более надежных и предсказуемых систем искусственного интеллекта, способных корректно функционировать в условиях реального мира. 