\newpage
\section*{АННОТАЦИЯ}
\thispagestyle{empty}

В данной работе проведено исследование проблемы пространственной инвариантности в сверточных нейронных сетях (CNN) и её влияния на стабильность работы алгоритмов компьютерного зрения. Особое внимание уделено артефактам, возникающим при субпиксельных сдвигах входных изображений, которые могут существенно влиять на качество классификации и детекции объектов.

В теоретической части работы формализована проблема отсутствия полной инвариантности к сдвигам в современных CNN-архитектурах, проанализированы причины этого явления, связанные с операциями субдискретизации (даунсэмплинга), и рассмотрены существующие подходы к её решению, включая методы анти-алиасинга и полифазной выборки.

Экспериментальная часть исследования сфокусирована на сравнительном анализе стандартных архитектур (ResNet-50, VGG-16, YOLOv5) и их модифицированных версий с различными методами обеспечения инвариантности к сдвигам. Разработана методология тестирования, включающая генерацию последовательностей изображений с субпиксельными сдвигами объектов и комплексную систему метрик для оценки стабильности.

Результаты экспериментов демонстрируют, что стандартные CNN-архитектуры проявляют значительную нестабильность даже при минимальных сдвигах входных данных. Применение методов анти-алиасинга (BlurPool) существенно улучшает стабильность, а наилучшие результаты показывает внедрение техники полифазной выборки (TIPS), которая почти полностью устраняет артефакты пространственной вариативности при небольшом увеличении вычислительной сложности.

На основе проведенного исследования сформулированы практические рекомендации по выбору архитектур и методов обеспечения инвариантности к сдвигам для различных задач компьютерного зрения, что может быть полезно при разработке систем, требующих высокой точности и стабильности работы.

\bigskip
\textbf{Ключевые слова:} сверточные нейронные сети, пространственная инвариантность, анти-алиасинг, BlurPool, TIPS, компьютерное зрение, YOLOv5.

\newpage