\newpage
\section*{АННОТАЦИЯ}
\thispagestyle{empty}

В данной работе проведено исследование проблемы пространственной инвариантности (shift-invariance) в сверточных нейронных сетях (CNN) и её влияния на стабильность работы алгоритмов компьютерного зрения. Особое внимание уделено артефактам, возникающим при субпиксельных сдвигах входных изображений, которые могут приводить к значительным изменениям в выходных данных модели, снижая надежность систем классификации и детекции объектов.

В теоретической части работы формализована проблема отсутствия полной инвариантности к сдвигам в современных CNN-архитектурах, проанализированы фундаментальные причины этого явления, связанные с операциями субдискретизации (даунсэмплинга) и нарушением теоремы Найквиста-Шеннона. Предложена математическая модель, описывающая возникновение эффекта алиасинга при операциях пулинга и страйдинга, и рассмотрены существующие подходы к его устранению.

Экспериментальная часть исследования сфокусирована на сравнительном анализе стандартных архитектур (ResNet-50, VGG-16, YOLOv5) и их модифицированных версий с различными методами обеспечения инвариантности к сдвигам. Разработана методология тестирования, включающая генерацию последовательностей изображений с контролируемыми субпиксельными сдвигами объектов и комплексную систему метрик для количественной оценки стабильности. Предложена и реализована собственная метрика для измерения вариативности выходных данных моделей при малых изменениях входа.

Результаты экспериментов демонстрируют, что стандартные CNN-архитектуры проявляют значительную нестабильность (до 30\% вариации в оценке вероятности класса и до 25\% в точности детекции) даже при субпиксельных сдвигах входных данных. Применение методов анти-алиасинга (BlurPool) снижает вариативность на 40-60\%, а разработанная в рамках исследования техника полифазной выборки (TIPS) позволяет сократить эффект вариативности на 85-95\% при увеличении вычислительной сложности всего на 5-8\%.)

На основе проведенного исследования сформулированы практические рекомендации по выбору архитектур и методов обеспечения инвариантности к сдвигам для различных задач компьютерного зрения, что особенно важно для критических приложений, где стабильность и предсказуемость работы моделей имеют первостепенное значение.

\bigskip
\textbf{Ключевые слова:} сверточные нейронные сети, пространственная инвариантность, shift-invariance, анти-алиасинг, BlurPool, TIPS, полифазная выборка, компьютерное зрение, YOLOv5.

\newpage