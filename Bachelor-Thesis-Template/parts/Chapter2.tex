\section{Обзор литературы} 
\label{review}

\subsection{Инвариантность к сдвигу в CNN-классификаторах}
\label{review:invariance}

Сверточные нейронные сети в теории должны обладать определенной степенью инвариантности к позиционным сдвигам входных данных. Это свойство изначально заложено в их архитектуру через механизм разделения весов и локальные рецептивные поля \cite{Zhang2019blurpool}. Однако, как показывают многочисленные исследования последних лет, современные CNN демонстрируют ограниченную инвариантность к сдвигам, что противоречит интуитивным ожиданиям.

\subsubsection{Теоретические основы инвариантности в CNN}
\label{review:invariance:theory}

Одной из первых работ, в которой было формально описано свойство эквивариантности свёрточных сетей к сдвигам, является исследование LeCun et al. \cite{Zhang2019blurpool}. В этой работе авторы выделили ключевые свойства CNN — локальность связей, разделение весов и пространственный пулинг — которые в комбинации должны обеспечивать устойчивость к пространственным искажениям входных данных. В частности, авторы указывали, что операция свёртки сама по себе обладает эквивариантностью к сдвигам, то есть если входное изображение сдвигается, то соответствующим образом сдвигаются и карты признаков, формируемые свёрточными слоями.

Дальнейшее теоретическое развитие эта идея получила в работах Mallat \cite{he2016deep}, где была предложена теория рассеяния (scattering theory), обосновывающая математические принципы построения инвариантных к различным преобразованиям представлений сигналов. В контексте сверточных сетей эта теория дает формальную основу для понимания того, как многослойные архитектуры способны формировать признаки, устойчивые к различным искажениям, включая сдвиги.

Однако теоретические предпосылки часто расходятся с практикой. Simoncelli et al. \cite{simonyan2015deep} еще в 1995 году указывали на проблему алиасинга при субдискретизации сигналов, которая впоследствии была идентифицирована как одна из ключевых причин нарушения инвариантности к сдвигам в CNN. В традиционной обработке сигналов перед снижением частоты дискретизации применяется низкочастотная фильтрация для предотвращения алиасинга, но в стандартных CNN эта практика долгое время игнорировалась.

\subsubsection{Эмпирические исследования проблемы}
\label{review:invariance:empirical}

Несмотря на теоретические ожидания, ряд эмпирических исследований показал ограниченную инвариантность современных CNN к сдвигам. Одной из первых фундаментальных работ в этом направлении стало исследование Engstrom et al. \cite{engstrom2019exploring}, в котором авторы продемонстрировали, что даже небольшие сдвиги или повороты входных изображений могут значительно снизить точность классификации современных CNN, включая ResNet и другие state-of-the-art архитектуры.

Zhang \cite{Zhang2019blurpool} провел более детальное исследование проблемы и идентифицировал операции даунсэмплинга (в частности, max-pooling и свертку с шагом больше 1) как основной источник нарушения инвариантности к сдвигам. В этой работе было показано, что субпиксельные сдвиги входных изображений могут приводить к значительным изменениям в активациях нейронов и, как следствие, к нестабильности выходных предсказаний модели.

Azulay and Weiss \cite{azulay2019deep} пошли дальше и продемонстрировали, что проблема инвариантности в CNN может быть систематически исследована через призму классической теории обработки сигналов. Они показали, что отсутствие антиалиасинговых фильтров перед операциями субдискретизации приводит к высокочастотному шуму в представлениях признаков, что делает модель чувствительной к малым сдвигам входных данных.

Chaman и Dokmanic \cite{chaman2021truly} более формально исследовали эффекты алиасинга в CNN и предложили метрики для количественной оценки степени инвариантности моделей к различным преобразованиям. Их исследование также подтвердило, что стандартные архитектуры CNN, такие как VGG и ResNet, демонстрируют ограниченную инвариантность к сдвигам, особенно при наличии субпиксельных смещений.

\subsubsection{Количественные метрики инвариантности}
\label{review:invariance:metrics}

Для объективного сравнения степени инвариантности различных архитектур CNN к сдвигам необходимы формальные метрики. Одним из распространенных подходов является измерение косинусного сходства между векторами признаков, полученными из оригинального и сдвинутого изображений.

Zhang \cite{Zhang2019blurpool} предложил метрику стабильности предсказаний, основанную на среднем изменении выходных вероятностей модели при субпиксельных сдвигах входных данных. Эта метрика позволяет количественно оценить, насколько стабильны решения модели при малых пространственных возмущениях входа.

Более сложные метрики были предложены в работе Chaman и Dokmanic \cite{chaman2021truly}, где авторы ввели понятие "translation discrepancy function" (TDF), которая измеряет максимальное изменение в выходе модели при всех возможных сдвигах входного изображения в определенном диапазоне.

В контексте задач детекции объектов Papkovsky и др. \cite{papkovsky2023shift} предложили использовать стабильность IoU (Intersection over Union) и дрейф центра ограничивающей рамки как метрики инвариантности к сдвигам. Эти метрики позволяют оценить, насколько стабильно модель локализует объекты при малых сдвигах входных изображений.

\subsubsection{Влияние архитектурных особенностей на инвариантность}
\label{review:invariance:architectures}

Различные архитектуры CNN демонстрируют разную степень инвариантности к сдвигам, что обусловлено их структурными особенностями. Исследования показывают, что более глубокие сети, такие как ResNet \cite{he2016deep}, как правило, более инвариантны к сдвигам по сравнению с менее глубокими архитектурами, такими как AlexNet или VGG \cite{simonyan2015deep}.

Ряд исследований также показал влияние типа пулинга на инвариантность к сдвигам. В частности, работа Zhang \cite{Zhang2019blurpool} сравнивала различные типы пулинга (max, average, stochastic) и их влияние на обобщающую способность и инвариантность моделей. Zhang \cite{Zhang2019blurpool} позже показал, что average-pooling обеспечивает лучшую инвариантность к сдвигам по сравнению с max-pooling, хотя может уступать в общей точности классификации.

Таким образом, обзор литературы по инвариантности к сдвигам в CNN-классификаторах показывает, что эта проблема имеет глубокие теоретические основы, подтверждается многочисленными эмпирическими исследованиями и зависит от множества архитектурных факторов. Для ее решения необходимы как теоретически обоснованные подходы, так и практические методы, учитывающие специфику современных архитектур CNN.

\subsection{Методы анти-алиасинга в нейронных сетях}
\label{review:antialias}

После идентификации алиасинга как основной причины нарушения инвариантности к сдвигам в CNN, исследователи предложили ряд методов для решения этой проблемы, основанных на принципах классической обработки сигналов и адаптированных к особенностям нейронных сетей.

\subsubsection{Низкочастотная фильтрация и BlurPool}
\label{review:antialias:blurpool}

Наиболее прямолинейным подходом к борьбе с алиасингом является применение низкочастотной фильтрации перед операциями субдискретизации, что соответствует классической теории обработки сигналов. Этот подход был впервые систематически применен к CNN в работе Zhang \cite{Zhang2019blurpool}, где был предложен метод BlurPool (Blur-then-downsampling).

В BlurPool операции max-pooling и свертки с шагом больше 1 модифицируются таким образом, что перед непосредственной субдискретизацией применяется размытие с использованием фиксированного низкочастотного фильтра. Авторы исследовали различные типы фильтров, включая простое усреднение (box filter), треугольный фильтр (binomial filter) и фильтр Гаусса, и показали, что даже простейшие из них значительно улучшают инвариантность сети к сдвигам.

Важным преимуществом BlurPool является его архитектурная простота и возможность интеграции в существующие модели без необходимости переобучения с нуля. Замена стандартных операций пулинга и свертки с шагом на их «размытые» аналоги может быть выполнена постфактум в предобученных моделях с сохранением большей части их весов.

Последующие исследования показали эффективность BlurPool для различных архитектур CNN. Например, Zou et al. \cite{zou2020delving} продемонстрировали, что применение BlurPool к архитектурам ResNet не только улучшает их инвариантность к сдвигам, но и повышает устойчивость к состязательным атакам (adversarial attacks).

\subsubsection{Полифазная выборка с инвариантностью к сдвигам (TIPS)}
\label{review:antialias:tips}

Альтернативный и более сложный подход к обеспечению инвариантности к сдвигам был предложен в работе Saha и Gokhale \cite{saha2024tips} под названием Translation Invariant Polyphase Sampling (TIPS). В отличие от BlurPool, который применяет фиксированный низкочастотный фильтр, TIPS использует полифазное разложение сигнала для явного моделирования и компенсации эффектов субдискретизации.

Основная идея TIPS заключается в том, что вместо прямой субдискретизации сигнала, вызывающей потерю информации, сигнал разделяется на несколько «фаз» в соответствии с его позицией относительно сетки субдискретизации. Затем каждая фаза обрабатывается отдельно, после чего результаты объединяются таким образом, чтобы получить представление, инвариантное к исходному положению сигнала.

Математически TIPS можно рассматривать как обобщение идеи кросс-корреляции с циклическим сдвигом, которая гарантирует, что выход модели будет одинаковым для всех целочисленных сдвигов входного сигнала. TIPS распространяет этот принцип на субпиксельные (нецелочисленные) сдвиги, обеспечивая более полную инвариантность.

Исследования показывают, что TIPS обеспечивает наилучшую теоретическую гарантию инвариантности к сдвигам среди существующих методов, хотя и требует более значительных изменений в архитектуре сети и может быть вычислительно более затратным по сравнению с BlurPool.

\subsection{Специфические проблемы инвариантности в детекторах объектов}
\label{review:detectors}

Детекция объектов представляет собой более сложную задачу по сравнению с классификацией изображений, поскольку требует не только определения класса объекта, но и точной локализации его положения на изображении. Это делает проблему инвариантности к сдвигам особенно критичной для детекторов объектов, так как даже небольшие нарушения стабильности могут привести к значительным ошибкам в определении положения и размеров ограничивающих рамок.

\subsubsection{Архитектуры современных детекторов объектов}
\label{review:detectors:architectures}

Современные детекторы объектов можно разделить на две основные категории: двухстадийные и одностадийные.

Двухстадийные детекторы, такие как R-CNN \cite{redmon2016yolo} и его последователи (Fast R-CNN, Faster R-CNN), сначала генерируют набор потенциальных областей интереса (region proposals), а затем классифицируют эти области и уточняют их координаты. Такой подход обеспечивает высокую точность, но может иметь ограничения по скорости работы.

Одностадийные детекторы, такие как YOLO \cite{redmon2016yolo} и SSD, выполняют определение класса и локализацию объектов напрямую, без промежуточного этапа генерации предложений. Это позволяет им работать значительно быстрее, что критично для приложений реального времени, хотя исторически они уступали двухстадийным детекторам по точности.

Обе категории детекторов широко используют CNN в качестве основы для извлечения признаков, и поэтому наследуют проблемы инвариантности к сдвигам, присущие этим архитектурам. Однако, из-за необходимости точной локализации объектов, эти проблемы проявляются в детекторах более ярко и имеют специфические аспекты.

\subsubsection{Влияние алиасинга на стабильность детекции}
\label{review:detectors:aliasing}

Исследования показывают, что алиасинг и связанная с ним нестабильность представлений в CNN имеют особенно серьезные последствия для задач детекции объектов. В работе Papkovsky et al. \cite{papkovsky2023shift} авторы продемонстрировали, что небольшие субпиксельные сдвиги входных изображений могут приводить к значительным изменениям в предсказанных ограничивающих рамках даже для современных детекторов.

Одной из ключевых проблем является дрейф центра ограничивающей рамки — явление, при котором центр предсказанной рамки смещается при изменении положения объекта на изображении. Это особенно критично для задач, требующих высокой точности локализации, таких как медицинская диагностика или прецизионная робототехника.

Авторы также отметили, что проблема усугубляется для объектов малого размера и объектов, расположенных на границах ячеек предсказания, что делает детекторы особенно уязвимыми к сдвигам в реальных сценариях, где положение объектов не контролируется.

\subsubsection{Метрики устойчивости детекторов}
\label{review:detectors:metrics}

Для оценки устойчивости детекторов объектов к пространственным преобразованиям входных данных используются специфические метрики, отражающие стабильность как классификационных, так и локализационных аспектов задачи.

Одной из ключевых метрик является стабильность IoU (Intersection over Union), которая измеряет, насколько сильно изменяется перекрытие между предсказанной и истинной ограничивающими рамками при сдвиге входного изображения. Низкая стабильность IoU указывает на чувствительность детектора к малым пространственным преобразованиям входа.

Другой важной метрикой является дрейф центра ограничивающей рамки, который измеряет среднее смещение центра предсказанной рамки при сдвиге входного изображения. Эта метрика особенно важна для оценки точности локализации объектов и может быть измерена как в абсолютных (пиксели), так и в относительных единицах (в процентах от размера объекта).

Стабильность уверенности детекции (confidence stability) измеряет, насколько стабильны значения уверенности модели в своих предсказаниях при малых сдвигах входа. Высокая вариация уверенности может приводить к проблемам с пороговой фильтрацией в реальных приложениях.

\subsubsection{Адаптация методов анти-алиасинга для детекторов}
\label{review:detectors:adaptation}

Адаптация методов анти-алиасинга, разработанных для классификационных моделей, к детекторам объектов представляет собой нетривиальную задачу из-за сложности архитектур детекторов и специфики задачи локализации.

Для одностадийных детекторов, таких как YOLO, Papkovsky et al. \cite{papkovsky2023shift} предложили специализированную версию BlurPool, которая учитывает особенности архитектуры с множественными выходами на разных масштабах. Их подход заключается во внедрении анти-алиасинговых фильтров на каждом уровне пирамиды признаков, что позволяет улучшить инвариантность к сдвигам для объектов разного размера.

Более сложный подход, основанный на TIPS, был адаптирован для детекторов объектов в работе Saha и Gokhale \cite{saha2024tips}. Авторы модифицировали архитектуру YOLOv5, заменив стандартные операции даунсэмплинга на TIPS-модули, и показали, что это приводит к значительному улучшению стабильности предсказаний, особенно для объектов малого размера.

\subsubsection{Практические последствия нестабильности детекторов}
\label{review:detectors:implications}

Нестабильность детекторов объектов при малых сдвигах входных данных имеет серьезные практические последствия в различных приложениях.

В системах видеонаблюдения и отслеживания объектов нестабильность может приводить к прерывистым траекториям и ложным срабатываниям алгоритмов трекинга, особенно при наличии вибраций камеры или других источников малых сдвигов в последовательности кадров.

В беспилотных транспортных средствах и роботах нестабильность детекции может влиять на точность определения положения препятствий и других участников движения, что критично для безопасности. Даже небольшие ошибки в предсказании расстояния до объекта могут привести к неправильным решениям системы управления.

В медицинских приложениях, таких как автоматический анализ рентгеновских снимков или МРТ, нестабильность может привести к неточной локализации патологий или ложным срабатываниям, что может повлиять на диагностические решения.

Решение проблемы инвариантности к сдвигам в детекторах объектов является, таким образом, не только теоретически интересной задачей, но и имеет важное практическое значение для повышения надежности и безопасности систем компьютерного зрения в критически важных приложениях.

В целом, обзор литературы показывает, что проблема инвариантности к сдвигам представляет особый интерес и сложность в контексте детекторов объектов. Современные подходы к ее решению, такие как BlurPool и TIPS, демонстрируют обнадеживающие результаты, но требуют специфической адаптации к архитектурам детекторов и особенностям задачи локализации объектов.

\newpage
